\documentclass[main.tex]{subfiles}


\begin{document}

\chapter{Conclusion}

In this thesis, we discuss aspects of the decision process, in
particular, the mathematical formulation of optimisation problems and
algorithms used by numerical solvers to approximate
solutions to such problems.
When mathematicians and practitioners use models with randomness, they
often marginalise over random variables and then mostly ignore the
full distribution of the outcomes. The argument is often that the
marginalised value is a good representation of what the decision maker
cares, or should care, about.
It is the intention of this thesis to highlight that the distribution
of the outcomes for relevant decisions can both guide the choice of
optimisation formulation and provide information regarding the impact
algorithm decisions have on modelled risk-attitudes.
The examples we provide to motivate the work are mainly toy-problems
related to pricing of retail products, however, the considerations
apply equally to other decision processes with random outcomes.

\Cref{ch:onestage} illustrates that when the distributions are
uni-modal, or approximately normal, then formulating the optimisation
problem using mean and standard deviation can yield the same decision
as using super-quantiles. This is advantageous from both an
implementational and a computational point of view.
\Cref{ch:discrete_control,ch:cts_control} provide further examples of
how simplified formulations that ignore the randomness result in
marginalised outcomes that are practically equal to the values from
decisions based on the original formulation. A further investigation
of the distribution of outcomes shows that the simplified formulations
result in the best outcomes for half of the realisations of the
underlying random variables. In the remaining realisations, they result
in much worse outcomes, which we interpret as an algorithmic decision
that has impacted the modelled risk-attitude.
The optimisation problems that we formulate in this thesis need to
be solved numerically using some optimisation routine using gradient
information. In \Cref{ch:objaccel} we propose a new algorithm that can
be combined with existing algorithms and we show, with several
numerical experiments, that it reduces the computational cost of
finding a solution with a prescribed accuracy.

The examples that we use to illustrate the
impact of decisions and distributions of outcomes are simplified
problems of retail applications. They provide a template for types
of investigations that can help the development of decision making
infrastructure in real-world applications.
In our examples, these investigations conclude that simplified
formulations reduce computational cost with little impact on the
objective, however, more complicated systems may behave differently.
A natural avenue for further
research is to consider a real-world system where the demand models
and the distributional information is provided from a retailer. Within
this system one could then:
\begin{enumerate}
\item Investigate the typical distributions of objectives that
  arise based on this data;
\item propose two to four optimisation formulations; and
\item compare the computational cost, decisions, and distribution of
  outcomes that arise from the different formulations.
\end{enumerate}

The presence of competitors or external decisions that we do not
directly control is not discussed in this thesis. Competition is believed to
have a significant impact on the objectives by decision maker's in
business, and work to better understand such game theoretic matters
are important. A difficulty with theoretical investigations of
such situations is to that the researchers have too much control over
the problem: they decide apriori both how the underlying system
behaves and what policies the competitors can choose from.  We have
contributed to an
experiment that runs a pricing and
modelling competition between several competitors
\citep{geer2018dynamic}.  Each competitor writes an algorithm that,
only based on data, dynamically prices a product over \num{1000}
periods to maximise revenue without knowing apriori what algorithms
the competitors use. We believe this provides a good way to test how
the algorithms for decision-making behaves in unexpected environments,
and recommend that similar competitions are run within the retailer
before a new decision-making system goes ``live''.



\biblio{} % Bibliography when standalone
\end{document}

%%% Local Variables:
%%% mode: latex
%%% TeX-master: t
%%% TeX-command-extra-options: "--shell-escape"
%%% End:
