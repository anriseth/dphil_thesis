\documentclass[main.tex]{subfiles}


\begin{document}

\chapter{Conclusion}

In this thesis we discuss aspects of the decision process, in
particular, the mathematical formulation of optimisation problems and
algorithms used by numerical solvers to approximate
solutions to such problems.
When mathematicians and practitioners use models with randomness, they
often marginalise over random variables and then mostly ignore the
full distribution of the outcomes. The argument is often that the
marginalised value is a good representation of what the decision maker
cares, or should care, about.
It is the intention of this thesis to highlight that the distribution
of the outcomes for relevant decisions can both guide the choice of
optimisation formulation and provide information regarding the impact
algorithm decisions have on modelled risk-attitudes.
The examples we provide to motivate the work are mainly toy-problems
related to pricing of retail products, however, the considerations
apply equally to other decision processes with random outcomes.

\Cref{ch:onestage} illustrates that when the distributions are
uni-modal, or approximately normal, then formulating the optimisation
problem using mean and standard deviation can yield the same decision
as using semi-quantiles. This is advantageous from both an
implementational and a computational point of view.
\Cref{ch:discrete_control,ch:cts_control} provide further examples of
how simplified formulations that ignore the randomness result in
marginalised outcomes that are practically equal to the values from
decisions based on the original formulation. A further investigation
of the distribution of outcomes show that the simplified formulations
result in the best outcomes for half of the realisations of the
underlying random variables. In the remaining realisations they result
in much worse outcomes, which we interpret as an algorithmic decision
that has impacted the modelled risk-attitude.


\todo[inline]{Optimisation algorithm}


Further work
\begin{itemize}
\item Each chapter provides some suggestions for further work.
\item Main thing --- use data and models from a real retailer to see
  whether the same conclusions carry over to larger problems.
\end{itemize}

\biblio{} % Bibliography when standalone
\end{document}

%%% Local Variables:
%%% mode: latex
%%% TeX-master: t
%%% TeX-command-extra-options: "--shell-escape"
%%% End:
