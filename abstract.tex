\documentclass[main.tex]{subfiles}

\begin{document}
We all want to make good decisions. The decision
process requires understanding and prediction of systems that can
involve complex interactions fraught with uncertainty. It is, therefore,
the ultimate challenge for mathematics to come up with good practices
to guide decision makers.  In this thesis we will investigate
algorithms for different steps in the decision process, focusing on
systems where we are uncertain about the outcomes but can quantify how
probable they are using random variables.  Any decision one makes in
such a situation leads to a distribution of outcomes and requires a way
to evaluate a decision.  The standard approach is to
marginalise the distribution of outcomes into a single number that
tries to summarise the value of each decision.  We will discuss three
families of such methods; expected utilities, mean-deviation methods,
and nonlinear expectations. After selecting a marginalisation
approach, mathematicians and decision makers alike
focus their analysis on the marginalised value but ignore the
distribution. This thesis argues that we should also be investigating
the implications of the chosen mathematical approach for the whole
distribution of outcomes.

We illustrate how to gain insight into the effect different mathematical
formulations have on the distribution with one-stage and sequential
decision problems in discrete and continuous time. Product pricing
decisions in retail motivate our work, however, the insights
apply equally to a wide range of decision problems.  We show that
different ways to marginalise the distributions can result in the same
decisions but each way has a different complexity and computational
cost. Sequential decision problems are often computationally
intractable to solve to optimality and much research goes into
developing algorithms that are suboptimal in the marginalised sense,
but work within the computational budget available. If the performance
of these algorithms is evaluated they are mainly judged based on the
marginalised values, however, comparing the performance using the full
distribution provides interesting information: We provide numerical
examples from dynamic pricing applications where the suboptimal
algorithm results in higher profit than the optimal algorithm in more
than half of the realisations, which is paid for with a more significant
underperformance in the remaining realisations.

All the problems discussed in this thesis lead to continuous
optimisation problems. We develop a new algorithm that can be used on
top of existing optimisation algorithms to reduce the cost of
approximating solutions. The algorithm is tested on a range of
optimisation problems and is shown to be competitive with existing
methods.
\thispagestyle{empty}
\end{document}

%%% Local Variables:
%%% mode: latex
%%% TeX-master: t
%%% End:
