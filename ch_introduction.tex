\documentclass[main.tex]{subfiles}


\begin{document}

\chapter{Introduction}
This doctoral thesis presents a collection of topics that are relevant
in decision making under uncertainty, primarily motivated by
problems in the retail industry.
We will speak primarily about the decision process involved in setting the
price of products, including demand modelling, estimation from data,
handling of uncertainty, and optimization algorithms.

The process of pricing products in order to control demand and
maximize revenues has been undertaken for centuries. In recent
decades, data- and model-driven approaches have become increasingly
popular in order to advise on and automate the process for companies.
There are several success stories from early adopters, for example in
the airline industry.
American Airlines estimated in 1992 that the introduction of revenue
management software had, over the preceding three
years, contributed \SI{500}[\$] million of additional revenue per year,
and would continue to do so in the future~\cite{smith1992yield}.
In an example from Chilean retail, the authors of~\cite{bitran1998coordinating}
report an expected revenue improvement of \SIrange{7}{16}{\percent} after implementing
model-driven strategies.
This range of financial impact due to implementation of pricing and revenue
management systems are further supported in other studies, see \cite[Ch.~1.2]{phillips2005pricing}.
For retailers with billions of pounds in revenues, small
improvements to their revenue-management processes can be worth millions
of pounds.
In addition unsold items add up to thousands of tonnes of waste per year, so
improving the control of demand for products is advantageous
for both retailers and the environment.

The theory and practice of pricing and revenue management is highly
multi-disciplinary, and attracts research from a wide variety of
fields. Our focus will be on different aspects of the modelling and
optimization procedures, from a more mathematical point of view.
\citet{phillips2005pricing} and \citet{ozer2012oxford} contain perspectives on the
revenue management process prevalent in the social sciences.
A notable resource with more mathematical focus, which
cover both theoretical and practical
perspectives, is that of those of \citet{talluri2006theory}.
Other approaches to pricing of retail products more akin to the
mathematical modelling community can, for example, be found in \cite{butler2014customer}.

\section{Outline of the thesis}

\todo[inline]{Write introduction}

\section{Making decisions}

\begin{itemize}
\item Objective
\item Model of system
\item Data assimilation
\item Uncertainty: risk attitude/preferences
\item Optimization process: approximations and algorithms
\end{itemize}


\section{Modelling demand}
\begin{itemize}
\item Describe models (Talluri + my own)
\item Parameter fitting
\item Description of parameters as random variables
\end{itemize}

\section{Probability and random variables}
\begin{itemize}
\item Describe random variable, prob space, expectation, variance
\item Stochastic processes
\end{itemize}

\section{Optimization}

\begin{itemize}
\item Define in terms of minimization --- show maximization
\item Constraints
\item Primer on unconstrained optimization: First-order condition
\item Steepest descent, Newton, Quasi-Newton
\item Constraint conditions: KKT et.\ al.
\end{itemize}

\biblio{} % Bibliography when standalone
\end{document}

%%% Local Variables:
%%% mode: latex
%%% TeX-master: t
%%% TeX-command-extra-options: "--shell-escape"
%%% End:
