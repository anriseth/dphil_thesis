\documentclass[main.tex]{subfiles}


\begin{document}

\chapter{Objective acceleration for unconstrained optimisation}\label{ch:objaccel}

We now move from modelling decision making processes
to developing an algorithm that a computer
can use to approximate the solution to an optimisation problem
\begin{equation}
  \min_{x\in\mathbb{R}^n} \Obj(x).
\end{equation}
A local minimum $x^*$ of $\Obj$ is one such that
$\Obj(x^*)\leq \Obj(x)$ for all $x$ in a neighbourhood of $x^*$.  When
$\Obj$ is sufficiently smooth, the gradient at $x^*$ is zero,
$\nabla\Obj(x^*)=0$.  Derivative based algorithms for minimising
$\Obj$ are mainly iterative procedures that look for roots of
$\nabla\Obj$.  Note that $\nabla\Obj(x)=0$ at local maximisers or
saddle points $x$ as well. To find a local minimum, the algorithms also
attempt to move in directions that decrease the value of
$\Obj$.  Two common examples of such algorithms are the steepest
descent and Newton methods \citep{nocedal2006numerical}.  The gradient
represents the direction where $\Obj$ increases the most, so the
negative gradient is the direction of steepest descent. Newton's
method minimises a second-order Taylor approximation to the objective.
Let the Hessian of $\Obj$ at $x$ be denoted by $\Hess{}(x)$.  Given a
current guess $x$ of a local minimiser of $\Obj$, the two algorithms
propose a new guess $x^P$ according to
\begin{align}\label{eq:steepest_descent_step}
  x^P(\lambda)
  &= x - \lambda \nabla \Obj(x)
  &\textnormal{(steepest descent)}&,\\
  x^P(\lambda)
  &= x-\lambda {\Hess{}(x)}^{-1}\nabla \Obj(x)
  &\textnormal{(Newton's method)}&,\label{eq:newton_step}
\end{align}
where $\lambda>0$ is determined using a line search algorithm.
Line search algorithms decide the step length for each iteration by
approximately solving the one-dimensional optimisation problem
\begin{equation}
  \min_{\lambda>0} \Obj(x^P(\lambda)).
\end{equation}

Newton's method has superior convergence properties to the steepest
descent algorithm, but the cost of calculating the
Hessian and solving the linear system may be prohibitive in some applications.  To find a middle
ground one can also try to approximate the inverse Hessian, known as
quasi-Newton methods, or systematically cover the search space,
as nonlinear conjugate gradient methods do.  Further information on
iterative optimisation algorithms and line search procedures is covered by
\citet{nocedal2006numerical}.  In this chapter, we focus on a
class of algorithms that are used in combination with other algorithms
to, somehow, accelerate the guess $x^P$ towards a local minimiser.  We
refer to them as acceleration methods.  In most of the work on these
algorithms, such as the nonlinear generalised minimal residual method
(N-GMRES), acceleration is based on minimising the $\ell_2$ norm of
some target subspaces of $\mathbb{R}^n$.  We propose a natural
modification to N-GMRES, which significantly improves the performance
in a testing environment originally used to advocate N-GMRES.  Our
proposed approach, which we refer to as O-ACCEL (Objective
acceleration), is novel in that it minimises an approximation to the
objective function on subspaces of $\mathbb{R}^n$.


The N-GMRES and O-ACCEL algorithms have been implemented in
Optim \citep{mogensen2018optim} as part of the thesis work. In this
chapter we discuss the algorithm in the context of unconstrained
optimisation. The optimisation problems carried out in
\Cref{ch:onestage,ch:discrete_control,ch:cts_control} are all
constrained so that the decision variable lies in a compact box in
$\mathbb{R}^n$. We can still use the algorithm proposed here for these
problems as an inner optimiser of the box-constrained optimisation
procedure \texttt{Fminbox} in Optim.  The work which this chapter
is based on
% has been submitted to ``Numerical Linear Algebra with
% Applications''.\todo{Update with publication} It
originated
from an investigation of N-GMRES for nonlinear solvers in reservoir
simulation~\citep{riseth2015nonlinear}.


\section{Acceleration methods overview}
Gradient based optimisation algorithms normally iterate based on
tractable approximations to the objective function at a particular
point.
% The more sophisticated algorithms may improve their approximations
% based on previous iterates.
Acceleration algorithms aim to combine the strengths of existing
solvers with information from previous iterates.  We propose an
acceleration scheme that can be used on top of existing optimisation
algorithms, which generates a subspace from previous iterates, over
which it aims to optimise the objective function.
% We call the
% algorithm O-ACCEL, short for Objective Acceleration.

Our idea closely resembles the work of \citet{sterck2013steepest},
which introduced the preconditioned N-GMRES
algorithm for optimisation. By using a more appropriate target to
accelerate the optimisation than N-GMRES does, we show, with numerical
examples, how O-ACCEL more efficiently accelerates the steepest
descent algorithm.  When optimising an objective $\Obj$, N-GMRES is
used as an accelerator from the point of view of solving the nonlinear
system $\nabla\Obj(x)=0$ which arises from the first-order condition
of optimality.  It uses the idea of Krylov subspace acceleration from
\citet{washio1997krylov} and \citet{oosterlee2000krylov} for solving
nonlinear equations that arise from discretisations of partial
differential equations.  The name N-GMRES arises from the fact that
steepest descent preconditioned N-GMRES is equivalent to the standard
GMRES procedure for linear systems of equations
\citep{washio1997krylov,sterck2013steepest}.  A similar idea, also
arising from nonlinear equations, was described by
\citet{anderson1965iterative}. See
\citet{walker2011anderson} for a note on the similarities of the
methods, and \citet{fang2009two} which puts Anderson acceleration in
the context of a Broyden-type approximation of the inverse Jacobian.
% Other names, for similar acceleration methods used in computational
% chemistry, are direct inversion of the iterative subspace
% (DIIS)\cite{pulay1980convergence} due to
% \citeauthor{pulay1980convergence} and conjugate residual with
% optimal trial vectors (CROP)\cite{ettenhuber2015discarding}.
\citet{brune2015composing} show, with many numerical examples, that
N-GMRES and Anderson acceleration can greatly improve convergence on
nonlinear systems, when combined with an appropriate preconditioner
(nonlinear solver), by reducing the number of function and Jacobian
evaluations required to reach a given tolerance.
In the setting of optimisation,
\citet{sterck2012nonlinear} and \citet{sterck2016nonlinearly} show
large improvements in convergence by applying N-GMRES acceleration to
the computation of tensor decompositions.

More recently, \citet{damien2016regularized} have developed another
acceleration method for convex optimisation denoted regularised
nonlinear acceleration (RNA), which \citet{cartis2017accelerating}
have extended to the nonconvex case.  Acceleration techniques differ
from one another in several ways, but, for convex quadratic
objectives, the Anderson, N-GMRES and Scieur et al.\ algorithms all
coincide \citep{cartis2017accelerating}.  These methods all minimise
the $\ell_2$ norm of some objective in $\mathbb{R}^n$, the space of
the decision variable.  The proposed algorithm in this chapter
instead aims to minimise the objective function over a subspace of
$\mathbb{R}^n$.  We believe this is a natural target to accelerate
against, especially when the optimisation procedure is seeking descent
directions.  For convex, quadratic functions we prove that O-ACCEL
with a steepest descent preconditioner reduces to the full
orthogonalisation method (FOM,~\citet{saad2003iterative}), a Krylov
subspace procedure for solving linear systems. This differentiates our
method from the other acceleration techniques, which are related to
the GMRES algorithm for linear systems.

Due to the close similarity with the proposed algorithm and N-GMRES,
this chapter focuses on numerical comparisons to N-GMRES under the
same testing conditions as used by \citet{sterck2013steepest}. On the
test set from \citet{sterck2013steepest}, our acceleration scheme
compares favourably to N-GMRES, as well as implementations of the
nonlinear conjugate gradient (N-CG) and limited-memory
Broyden--Fletcher--Goldfarb--Shanno (L-BFGS) methods
\citep{nocedal2006numerical}.  Further tests on the CUTEst test
problem set \citep{gould2015cutest} show that L-BFGS is more
applicable to these problems, however, O-ACCEL again performs better
than N-GMRES.

The chapter is organized as follows. Motivation for the algorithm,
and discussion around it, is covered in \Cref{sec:algo_description}.
Numerical tests that show the efficiency of our proposed acceleration
procedure applied to steepest descent are presented in
\Cref{sec:num_experiments}. We conclude and discuss further potential
work in \Cref{sec:accel_conclusion}.

\section{Optimisation acceleration with O-ACCEL}\label{sec:algo_description}
To fix notation, consider a twice continuously differentiable function
$\Obj\in C^2(\mathbb{R}^n)$ that is bounded below and has at least one
minimiser.  We aim to find a local minima of the optimisation problem
\begin{equation}
  \min_{x\in\mathbb{R}^n}\Obj(x).
\end{equation}
Let $\Precon(\Obj,x)$ denote an optimisation procedure for $\Obj$ with
initial guess $x\in\mathbb{R}^n$. This optimisation procedure can, for
example, be the application of one steepest descent, or Newton,
step~(\ref{eq:steepest_descent_step}, \ref{eq:newton_step}). We  refer to
$\Precon$ as the preconditioner, because it is
applied in the same fashion as a right preconditioner for iterative
procedures of linear systems \citep{brune2015composing}.  Given a
sequence of previously explored iterates $x^{(1)},\dots,x^{(k)}$, and
a proposed new guess $x^P = \Precon(\Obj,x^{(k)})$, we try to
accelerate the next iterate $x^{(k+1)}$ towards a minimiser.  Define
\begin{equation}
  \OKrylov{k}{x^P}=\mbox{span}\{x^{(1)}-x^P,\dots,x^{(k)}-x^P\}.
\end{equation}
The acceleration step aims to minimise $\Obj$ over the subset
$x^P+\OKrylov{k}{x^P}$, which can be interpreted as a generalisation
from a line search to a hyperplane search.  Let
$\alpha\in \mathbb{R}^k$, and set
\begin{equation}\label{eq:def_xA}
  x^A(\alpha) = x^P + \sum_{j=1}^k\alpha_j(x^{(j)}-x^P).
\end{equation}
Note that, when $k=1$, minimising $\Obj$ over $\OKrylov{k}{x^P}$ is
equivalent to the standard line search problem of minimising
$\lambda \mapsto \Obj(x^{(1)}+\lambda (x^P-x^{(1)}))$.  The
first-order condition for $\alpha$ to be a minimiser of the function
$\alpha\mapsto \Obj(x^A(\alpha))$ is
\begin{equation}
  \nabla_{\alpha} \Obj(x^P+{\textstyle\sum_{j=1}^k}\alpha_j(x^{(j)}-x^P)) =0.
\end{equation}
Define the gradient $\Grad(x)=\nabla_x\Obj(x)$. For $l=1,\dots,k$,
\begin{equation}\label{eq:deriv_objective_alpha}
  \frac{\partial }{\partial \alpha_l}
  \Obj({\textstyle x^P+\sum_{j=1}^k\alpha_j(x^{(j)}-x^P)})
  = \Transpose{{\Grad\left({\textstyle x^P+\sum_{j=1}^k\alpha_j(x^{(j)}-x^P)}\right)}}
  (x^{(l)}-x^P),
\end{equation}
where superscript $\intercal$ denotes the transpose.  The O-ACCEL
algorithm aims to linearise the first-order condition
$\nabla_\alpha \Obj(x^A(\alpha))=0$ in the following way. Let
$\Hess{}(x)$ denote the Hessian of $\Obj$ at $x$. By linearising
$\alpha\mapsto\nobreak\Grad(x^A(\alpha))$, we get
\begin{align}\label{eq:grad_xA_approx}
  \begin{split}
    \Grad\left({\textstyle
        x^P+\sum_{j=1}^k\alpha_j(x^{(j)}-x^P)}\right) &\approx
    \Grad(x^P) + \Hess{}(x^P)
    \sum_{j=1}^k\alpha_j(x^{(j)}-x^P)\\
    &= \Grad(x^P) +\Hess{}(x^P)(\vect{X}-\vect{X}^P)\alpha,
  \end{split}
\end{align}
where we use the matrices
$\vect{X} = \vecbracks{x^{(1)},\dots,x^{(k)}}\in\mathbb{R}^{n\times
  k}$ and
$\vect{X}^P = \vecbracks{x^P,\dots,x^P}\in\mathbb{R}^{n\times k}$.
Given this linearisation we aim to find an $\alpha\in\mathbb{R}^k$
that approximately satisfies the first-order
condition. % $\nabla_\alpha \Obj(x^A(\alpha))=0$.
We can do this by combining~\eqref{eq:deriv_objective_alpha}
and~\eqref{eq:grad_xA_approx}, and then look for an
$\alpha\in\mathbb{R}^k$ that solves
\begin{equation}
  \Transpose{\alpha}\Transpose{{(\vect{X}-\vect{X}^P)}}\Hess{}(x^P)
  (x^{(l)}-x^P) = -\Transpose{{\Grad(x^P)}}(x^{(l)}-x^P),\qquad l=1,\dots,k.
\end{equation}
In matrix form, the system of equations becomes
\begin{equation}\label{eq:ngmres_eq_hess}
  \Transpose{{\left(\vect{X}-\vect{X}^P\right)}}\Hess{}(x^P)
  \left(\vect{X}-\vect{X}^P\right)
  \alpha
  = -\Transpose{{(\vect{X}-\vect{X}^P)}}{\Grad(x^P)}.
\end{equation}

There are cases where we may not wish to compute the Hessian of $\Obj$
explicitly, for example, if $\Precon$ does not use it.  We can instead
use an approximation $\tilde{\Hess{}}(x^P)$ of the Hessian
$\Hess{}(x^P)$, or its action on vectors in $\OKrylov{k}{x^P}$.  The
iterative Hessian approximation algorithms that are used in
quasi-Newton methods can provide one avenue of research.  In the
numerical experiments provided in this chapter, we instead focus on
approximating the action of the Hessian on $\OKrylov{k}{x^P}$ to first
order by
\begin{equation}\label{eq:approx_hess}
  \Hess{}(x^P) (x^{(l)}-x^P)
  \approx \Grad(x^{(l)})-\Grad(x^P).
\end{equation}
Let
$\Grad(\vect{X}) = \vecbracks{\Grad(x^{(1)}),\dots,\Grad(x^{(k)})}$,
and define $\Grad(\vect{X}^P)$ similarly.  This gives a second
approximation to the first-order conditions,
\begin{equation}\label{eq:ngmres_eq_grad}
  \Transpose{{(\vect{X}-\vect{X}^P)}}\left(
    \Grad(\vect{X})-\Grad(\vect{X}^P)
  \right)
  \alpha
  = -\Transpose{{(\vect{X}-\vect{X}^P)}}{\Grad(x^P)}.
\end{equation}
In this chapter, we investigate the performance of the
objective-based acceleration using~\eqref{eq:ngmres_eq_grad}.

To contrast our work with the N-GMRES optimisation algorithm in
\citet{sterck2013steepest}, minimising the $\ell_2$ norm of the
approximation of $\Grad(x^A)$ established
from~\eqref{eq:grad_xA_approx} and~\eqref{eq:approx_hess} results in
the linear least squares problem
\begin{equation}
  \min_{\alpha\in\mathbb
    {R}^k}\Bigl\|\Grad(x^P)+\sum_{j=l}^k\alpha_l(\Grad(x^{(l)})-\Grad(x^P))\Bigr\|_2.
\end{equation}
Its solution can be found from the normal equation
\begin{equation}\label{eq:ngmres_lin_sys}
  \Transpose{{\left(\Grad(\vect{X})-\Grad(\vect{X}^P)\right)}}\left(
    \Grad(\vect{X})-\Grad(\vect{X}^P)
  \right)
  \alpha
  = -\Transpose{{\left(\Grad(\vect{X})-\Grad(\vect{X}^P)\right)}}{\Grad(x^P)}.
\end{equation}

We argue that the O-ACCEL algorithm is more appropriate for an
optimisation problem than N-GMRES. % When optimising over the whole
% decision space $\mathbb{R}^n$, it is natural to look for minimisers
% by solving $\Grad(x)=0$.
When we are restricted to subsets of the decision space, reduction in
the value of the objective is a better indicator of moving towards a
minimiser than reduction in the gradient norm. In effect, N-GMRES
ignores the extra information provided by $\Obj$.  This is better
illustrated in the case when $k=1$, where it is standard to perform a
line search on the objective rather than the gradient norm.


\subsection{Algorithm}
The proposed acceleration procedure, which we call O-ACCEL, is
described in \Cref{alg:ngmreso}.  The number of stored previous
iterates $w$ denotes the history size.  Setting an upper bound on the
history size can be necessary due to storage constraints, or to
prevent the local approximations of~\eqref{eq:grad_xA_approx}
and~\eqref{eq:approx_hess} from using iterates far away from $x^P$. If
the direction from $x^P$ to the accelerated step $x^A$
of~\eqref{eq:def_xA} is not a
descent direction, it indicates that the linearised approximation
around $x^P$ is bad for the currently stored iterates.  For
simplicity, we therefore choose to reset the history size to $w=1$
when we encounter such cases.
\begin{algorithm}[htb]
  \caption{The O-ACCEL algorithm}\label{alg:ngmreso}
  \begin{algorithmic}[1]
    \Procedure{OACCEL}{$x^{(1)},\dots,x^{(w)}$}
    \State $x^P\gets \Precon(\Obj,x^{(w)})$
    \State Approximate $\Hess{}(x^P)\approx \tilde{\Hess{}}$, or
    its action
    \State $A\gets
    \Transpose{{\left(\vect{X}-\vect{X}^P\right)}}\tilde{\Hess{}}
    \left(\vect{X}-\vect{X}^P\right)$
    \hfill {\footnotesize In \Cref{sec:num_experiments}: $A\gets\Transpose{{(\vect{X}-\vect{X}^P)}}\left(
        \Grad(\vect{X})-\Grad(\vect{X}^P)
      \right)$}
    \State $b\gets -\Transpose{{(\vect{X}-\vect{X}^P)}}{\Grad(x^P)}$
    \State Solve  $A\alpha = b$
    \State $x^A \gets x^P + \sum_{j=1}^w\alpha_j(x^{(j)}-x^P)$
    \If{$x^A-x^P$ is a descent direction}\label{algline:descent_dir}
    \State $x^{(w+1)}\gets \textnormal{linesearch}(x^P+\lambda(x^A-x^P))$\label{algli:xp_xa_ls}
    \State \texttt{reset} $\gets$ \texttt{false}
    \Else
    \State $x^{(w+1)}\gets x^P$
    \State \texttt{reset} $\gets$ \texttt{true}
    \EndIf
    \State \Return $(x^{(w+1)},\texttt{reset})$
    \EndProcedure
  \end{algorithmic}
\end{algorithm}

To prevent re-computation of $\Grad(x^{(j)})$ for $j=1,\dots,w$ in
each application of the procedure, we store these vectors for later
use.  The computational cost of the algorithm is approximately the
same as $w$-history L-BFGS with two-loop recursion
\citep{sterck2013steepest}. %.,
% see~\cite{sterck2013steepest}.
In terms of storage, O-ACCEL and L-BFGS both store $2w$ vectors of
size $n$. In addition, our implementation of O-ACCEL, as described in
\Cref{alg:ngmreso_impl} below, reduces the number of flops required by
storing a $w\times w$ matrix of previously calculated values. For the
numerical experiments we have used $w=20$, in accordance with
\citet{sterck2013steepest}.  It was, however, shown by
\citet{sterck2013steepest} that N-GMRES can already provide good
results with $w=3$.  Tests using O-ACCEL with $w=5$, although not
included here, provide almost as good results as reported in
\Cref{sec:num_experiments}.  Note that, if the Hessian is sparse, it
may be more storage efficient to find $\alpha$ from the linear system
in~\eqref{eq:ngmres_eq_hess} than using a large $w$.

\begin{remark}
  The RNA approach suggested by \citet{damien2016regularized} is
  called separately from the iterations by the optimiser, when judged
  appropriate. This contrasts with \Cref{alg:ngmreso}, where the
  acceleration happens at each iteration.  The O-ACCEL acceleration
  can be applied separately in the same fashion as in
  \citet{damien2016regularized}, however, this is not considered in
  this chapter.
\end{remark}

\subsection{O-ACCEL as a full orthogonalisation method (FOM)}
The optimality condition~\eqref{eq:deriv_objective_alpha} for the
function $\alpha\mapsto \Obj(x^A(\alpha))$ is
$\Transpose{\Grad(x^A)}(x^{(l)}-x^P)=0$, for $l=1,\dots,k$.  Hence, we
look for $x^A\in x^P+\OKrylov{k}{x^P}$ so that
$\Grad(x^A)\perp \OKrylov{k}{x^P}$.  This condition reduces to FOM
\citep{saad2003iterative} when $\Grad(x)$ is linear and
$\Precon(\Obj,x)$ is a steepest descent algorithm.  When the Hessian
is symmetric positive-definite, FOM is mathematically equivalent to
the conjugate gradient method. We can therefore think of O-ACCEL as a
N-CG method that approximates the orthogonality condition with a
larger history size.

The FOM is an iterative procedure for solving a linear system $Ax=b$.
With initial guess $x^{(1)}$ and residual $r^{(1)}=b-Ax^{(1)}$, define
the Krylov subspace
\begin{equation}
  \Krylov{k}=\mbox{span} \{r^{(1)},Ar^{(1)},\dots,A^{k-1}r^{(1)}\}.
\end{equation}
The FOM iterate $x^{(k+1)}$ is an element in $x^{(1)}+\Krylov{k}$
such that the residual is perpendicular to the Krylov subspace,
\begin{equation}
  b-Ax^{(k+1)}\perp \Krylov{k}.
\end{equation}

For convex, quadratic objectives
$\Obj(x)=\frac{1}{2}\Transpose{x}Ax-\Transpose{x}b$, the gradient
$\Grad(x)=Ax-b$ is linear and the optimum must satisfy the equation
$Ax=b$.  The residuals $r^{(k)}=b-Ax^{(k)}$ are equal to the negative
gradient $-\Grad(x^{(k)})$. Therefore, O-ACCEL with a steepest descent
preconditioner yields
$x^P = \Precon(\Obj,x^{(k)}) = x^{(k)}+\lambda^{(k)}r^{(k)}$ for some
$\lambda^{(k)}>0$.

\begin{theorem}\label{thm:oaccel_fom}
  Let $\Precon$ be a steepest descent preconditioner and
  $\Obj(x)=\frac{1}{2}\Transpose{x}Ax-\Transpose{x}b$.  Let the
  O-ACCEL algorithm take the step $x^{(w+1)}=x^A$ in
  \Cref{algli:xp_xa_ls} of \Cref{alg:ngmreso}.  Then the iterates of
  the O-ACCEL algorithm form the FOM sequence of the linear system
  $Ax=b$.
\end{theorem}

We shall shortly prove the theorem after deriving new expressions for
$\Krylov{k}$.  First, note that for any $x$, a reordering of terms can
show that
\begin{align}\label{eq:okrylov_2}
  \OKrylov{k}{x}
  &=
    \mbox{span}\{x-x^{(k)},x^{(k-1)},\dots,x^{(2)}-x^{(1)}\},\\
    % &\textnormal{and}\\
  x+\OKrylov{k}{x}
  &= x^{(1)} +
    \mbox{span}\{x-x^{(k)},x^{(k-1)},\dots,x^{(2)}-x^{(1)}\}.
    \label{eq:x_okrylov_2}
\end{align}
This motivates the next lemma, which connects the space on the right
hand side of~\eqref{eq:okrylov_2} to $\Krylov{k+1}$.
\begin{lemma}\label{lem:fom_xp}
  Let $x^{(1)},\dots,x^{(k)}$ be a given sequence of FOM iterates for
  a linear system $Ax=b$.  Assume that
  $\mbox{span}(x^{(k)}-x^{(k-1)},\dots,x^{(2)}-x^{(1)})=\Krylov{k}$,
  and let $x^P = x^{(k)}+\lambda r^{(k)}$ for some $\lambda>0$.  Then,
  \begin{equation}
    \mbox{span}\{x^P-x^{(k)},x^{(k)}-x^{(k-1)},\dots,x^{(2)}-x^{(1)}\}
    = \Krylov{k+1}.
  \end{equation}
\end{lemma}
\begin{proof}
  By definition of $x^P$ and the properties of the FOM sequence,
  \begin{equation}\label{eq:xp_xk_rk}
    x^{P}-x^{(k)} = \lambda r^{(k)}\perp \Krylov{k}.
  \end{equation}
  As $x^{(k)}\in x^{(1)}+\Krylov{k}$, we have $r^{(k)}\in\Krylov{k+1}$
  because
  \begin{align}
    r^{(k)} &\in b - A(x^{(1)} +\Krylov{k})
              = r^{(1)} - A\Krylov{k}
              \in \Krylov{k+1}.
  \end{align}
  Therefore, $\mbox{span}\{r^{(k)},\Krylov{k}\}=\Krylov{k+1}$.  This
  equality yields the result by replacing $r^{(k)}$ and $\Krylov{k}$
  with~\eqref{eq:xp_xk_rk} and
  $\mbox{span}\{x^{(k)}-x^{(k-1)},\dots,x^{(2)}-x^{(1)}\}$.
\end{proof}

\begin{proof}[Proof of \Cref{thm:oaccel_fom}]
  We prove the result by induction on the sequence
  $x^{(1)},\dots,x^{(k)}$ arising from the O-ACCEL algorithm.  Let
  $k=2$, then
  \begin{align}
    x^{(2)}
    &= x^P + \alpha^{(1)}(x^{(1)}-x^P)\\
    &= x^{(1)} + \lambda^{(1)}r^{(1)} -
      \alpha^{(1)}\lambda^{(1)}r^{(1)}
      \in x^{(1)} + \Krylov{1}.
  \end{align}
  and so $\mbox{span}\{x^{(2)}-x^{(1)}\} = \Krylov{1}$.
  From~\eqref{eq:deriv_objective_alpha} the residual
  $b-Ax^{(2)}\perp x^P-x^{(1)} = \lambda^{(k)}r^{(1)}$, and thus
  $x^{(2)}$ is the second FOM iterate. This establishes the base case
  for the induction proof.

  The inductive step follows from \Cref{lem:fom_xp} together
  with~\eqref{eq:okrylov_2} and~\eqref{eq:x_okrylov_2}, and hence
  proves that the O-ACCEL iterates are the FOM iterates for $Ax=b$.
\end{proof}

\begin{remark}
  The connection to the FOM differentiates O-ACCEL from N-GMRES,
  Anderson acceleration, and RNA, which reduce to GMRES for quadratic
  objectives.
\end{remark}


\section{Numerical experiments}\label{sec:num_experiments}
In order to investigate the performance of the proposed algorithm, we
implement it with two preconditioners $\Precon$. The first is steepest
descent with line search, and the second is steepest descent with a
fixed step length.  They are compared to the N-GMRES algorithm with
the same preconditioners, and implementations of the nonlinear
conjugate gradient (N-CG) variant with the Polak-Ribi\`{e}re update
formula, and the two-loop recursion version of the limited-memory
Broyden--Fletcher--Goldfarb--Shanno (L-BFGS) method
\citep{nocedal2006numerical}.  The test problems considered in
\Cref{subsec:test_problems_sterck,subsec:experiment_design,subsec:perf_prof,subsec:tensor_optim}
are the same eight problems that were used
in \citet{sterck2013steepest} to advocate N-GMRES. We also include
experiments from \num{33} CUTEst problems to further test the
applicability of the algorithms.  The results are presented in the
form of performance profiles, as introduced by
\citet{dolan2002benchmarking}, based on the number of
function/gradient evaluations.

The main focus of this chapter is to compare the performance of the
proposed algorithm to the N-GMRES algorithm. To this end, we have used
the MATLAB implementation of this algorithm, available
online.\footnote{\href{http://www.hansdesterck.net/Publications-by-topic/nonlinear-preconditioning-for-nonlinear-optimization}
  {hansdesterck.net/Publications-by-topic/nonlinear-preconditioning-for-nonlinear-optimization}}
The O-ACCEL implementation, and the rest of the code required to
generate the test result data, is also made available by the
author.\footnote{\url{https://github.com/anriseth/objective_accel_code}}
Our implementation of O-ACCEL follows the exact same steps, only
replacing the calculations needed to solve the N-GMRES system
in~\eqref{eq:ngmres_lin_sys} with those of the linear system
in~\eqref{eq:ngmres_eq_grad}.  The implementation is detailed in
\Cref{alg:ngmreso_impl}. It closely follows the instructions from
\citet{washio1997krylov}, including a regularisation for the linear
system.

The regularisation is used to prevent the direct linear solver for
finding $\alpha$ from crashing when $A$ is ill-conditioned or
singular,
which can happen if the vectors $g(x^{(k)})-g(x^P)$ are linearly
dependent.  Let $A\in\mathbb R^{w\times w}$ denote the system matrix
$\Transpose{{(\vect{X}-\vect{X}^P)}}\left(
  \Grad(\vect{X})-\Grad(\vect{X}^P) \right)$.  Then, for some
tolerance $\epsilon_0>0$, set
$\epsilon=\epsilon_0\cdot\max{\{A_{ii}\}}_{i=1}^w$. The $\max$ term is
used to scale the regularisation in accordance with the optimisation
problem.  With $I\in\mathbb{R}^{w\times w}$ the identity matrix, we
solve the linear problem
\begin{equation}
  (A+\epsilon I)\alpha = b,
\end{equation}
rather than the linear problem $A\alpha=b$ as defined in
\Cref{alg:ngmreso}.  This is a Tikhonov type
regularisation \citep{neumaier1998solving}, often employed to
regularise ill-conditioned problems.  \citet{washio1997krylov} shows
that the error in the resulting $\alpha$ is negligible for the N-GMRES
problem~\eqref{eq:ngmres_lin_sys} provided $\epsilon$ is much smaller
than the smallest non-zero eigenvalue of the system matrix.  The error
for the O-ACCEL system can be analysed within a general Tikhonov
regularisation framework, see, for example,
\citet{neumaier1998solving}. We do not investigate the impact of the
regularisation parameter further in this chapter, and use the value
$\epsilon_0=10^{-14}$ that was used in the N-GMRES code by
\citet{sterck2013steepest}.

\begin{algorithm}[htb]
  \caption{Implementation of O-ACCEL algorithm. Indentation and curly
    brackets denote scope.}\label{alg:ngmreso_impl}
  \hspace*{\algorithmicindent} \textbf{Input}:
  $\Obj$, $\Grad$, $\Precon$, $x$, $w_{\textnormal{max}}$, $\epsilon_0$, tolerance
  description\\
  \hspace*{\algorithmicindent} \textbf{Output}:
  $x$ satisfying tolerance description
  \begin{algorithmic}[1]
    \While{Not reached tolerance}
    \State $\texttt{x}_1 \gets x$ ; $\texttt{r}_1\gets \Grad(x)$ ; $\texttt{q}_{11}\gets
    \Transpose{x}\texttt{r}_1$
    \State $w\gets 1$ ; $k\gets 0$ ; \texttt{reset} $\gets$ \texttt{false}
    \While{\texttt{reset} is \texttt{false}}
    \State $k\gets k+1$
    \State $x \gets \Precon(\Obj,x)$ ; $r\gets \Grad(x)$
    \If{reached tolerance}
    \State \textbf{break}
    \EndIf
    \State $\eta \gets \Transpose{x}{r}$
    \State
    \textbf{for} $i=1,\dots,w$
    \{ $\xi^{(1)}_i\gets
    \Transpose{\texttt{x}_i}r$ ; $
    \xi^{(2)}_i\gets
    \Transpose{x}\texttt{r}_i$ ; $
    \texttt{b}_i\gets \eta-\xi^{(1)}_i$ \}
    \State
    \textbf{for} $i=1,\dots,w$ \{
    \textbf{for} $j = 1,\dots, w$
    \{ $\texttt{A}_{ij}\gets \texttt{q}_{ij}-\xi^{(1)}_i-\xi^{(2)}_j+\eta$ \} \}
    \State $\epsilon \gets \epsilon_0 \cdot \max\{\texttt{A}_{11,}\dots,\texttt{A}_{ww}\}$
    \State Solve
    $
    \begin{pmatrix}
      \texttt{A}_{11}+\epsilon&\cdots&\texttt{A}_{1w}\\
      \vdots&\ddots&\vdots\\
      \texttt{A}_{w1}&\cdots&\texttt{A}_{ww}+\epsilon
    \end{pmatrix}
    \begin{pmatrix}
      \alpha_1\\\vdots\\\alpha_w
    \end{pmatrix}
    = \begin{pmatrix}
      \texttt{b}_1\\\vdots\\\texttt{b}_w
    \end{pmatrix}$
    \State $x^A\gets  x +  \sum_{i=1}^w \alpha_i(\texttt{x}_i-x)$
    \State $d\gets x^A-x$
    \If{$\Transpose{d}r \geq 0$}
    \State \texttt{reset} $\gets$ \texttt{true}
    \Else
    \State $x\gets \textnormal{linesearch}(x+\lambda d)$
    \State $w\gets \min(w+1,w_{\textnormal{max}})$
    \State $j \gets (k \mod w_{\textnormal{max}}) + 1 $
    \State $\texttt{x}_j\gets x$
    \State $\texttt{r}_j\gets \Grad(x)$
    \State \textbf{for} $i=1,\dots,w$
    \{ $\texttt{q}_{ij}\gets \Transpose{\texttt{x}_i}\texttt{r}_j$ ; $\texttt{q}_{ji}\gets \Transpose{\texttt{x}_j}\texttt{r}_i$ \}
    \EndIf
    \EndWhile
    \EndWhile
  \end{algorithmic}
\end{algorithm}

For the remainder of the section, we present the test problems,
provide details for the parameter choices, and discuss the test
results.

\subsection{Test problems from \citeauthor{sterck2013steepest}}\label{subsec:test_problems_sterck}
We describe the seven test problems from \citet{sterck2013steepest}.
All the functions are defined as $\Obj:\mathbb{R}^n\to\mathbb{R}$, and
the matrices mentioned are all in $\mathbb{R}^{n\times n}$.

Problem A. Quadratic objective function with symmetric, positive
definite diagonal matrix $D$,
\begin{align}
  \begin{split}
    \Obj(x)&= {\textstyle\frac{1}{2}}\Transpose{{(x-x^*)}}D(x-x^*),\text{ where }\\
    D&=\mbox{diag}(1,2,\dots,n), \text{ and }\\
    x^*&=\vecbracks{1,\dots,1}.
  \end{split}
\end{align}
The minimiser $x^*$ of Problem A is unique, with $\Obj(x^*)=0$. The
gradient is given by $\Grad(x)=D(x-x^*)$.

Problem B. Problem A with paraboloid coordinate transformation,
\begin{align}
  \begin{split}
    \Obj(x)&= {\textstyle\frac{1}{2}}\Transpose{{y(x-x^*)}}Dy(x-x^*),\text{ where }\\
    D&=\mbox{diag}(1,2,\dots,n),\\
    x^*&=\vecbracks{1,\dots,1}, \text{ and }\\
    y_1(z)&=z_1 \text{ and } y_j(z)=z_j-10z_1^2 \quad (i=2,\dots,n).
  \end{split}
\end{align}
The minimiser is again $x^*$, with $\Obj(x^*)=0$. The gradient is
$\Grad(x) = Dy(x-x^*) - 20(x_1-x_1^*) \times \left(
  \sum_{j=2}^n{(Dy(x-x^*))}_j \right)
\Transpose{\vecbracks{1,0,\dots,0}}$.

Problem C. Problem B with a random nondiagonal matrix $T$ with
condition number $n$,
\begin{align}
  \begin{split}
    \Obj(x)&={\textstyle\frac{1}{2}}\Transpose{{y(x-x^*)}}Ty(x-x^*),\text{ where }\\
    x^*&=\vecbracks{1,\dots,1},\\
    y_1(z)&=z_1 \text{ and } y_j(z)=z_j-10z_1^2 \quad (i=2,\dots,n),
    \text { and}\\
    T&=Q\,\mbox{diag}(1,2,\dots,n)\,\Transpose{Q},
  \end{split}
\end{align}
where $Q$ is a random orthogonal matrix.  As in Problems A and B, the
minimiser is $x^*$ with $\Obj(x^*)=0$. The gradient is
$\Grad(x) = Ty(x-x^*) - 20(x_1-x_1^*) \times \left(
  \sum_{j=2}^n{(Ty(x-x^*))}_j \right)
\Transpose{\vecbracks{1,0,\dots,0}}$.

Problem D. Extended Rosenbrock function, Problem (21) from
\citet{more1981testing},
\begin{align}
  \begin{split}
    \Obj(x)&={\textstyle\frac{1}{2}}\sum_{j=1}^n {t_j(x)}^2,\text{ where $n$ is even,}\\
    t_j&=10(x_{j+1}-x_j^2)\qquad\text{($j$ odd), and}\\
    t_j&=1-x_{j-1}\qquad\qquad\text{($j$ even).}
  \end{split}
\end{align}
The unique minimum $\Obj(x^*)=0$ is attained at
$x^*=\vecbracks{1,\dots,1}$.  The derivative can be computed using
$g_k(x)=\sum_{j=1}^n t_j \frac{\partial t_j}{\partial x_k}$,
($k=1,\dots,n$).  Gradients for Problems E-G can be computed in
similar fashion.

Problem E. Extended Powell singular function, Problem (22) from
\citet{more1981testing},
\begin{align}
  \begin{split}
    \Obj(x)&={\textstyle\frac{1}{2}}\sum_{j=1}^n {t_j(x)}^2,\text{
      where $n$ is a multiple of
      4,}\\
    t_{4j-3} &= x_{4j-3}+10x_{4j-2},\\
    t_{4j-2} &= \sqrt{5}(x_{4j-1}-x_{4j}),\\
    t_{4j-1} &= {(x_{4j-2}-2x_{4j-1})}^2,\\
    t_{4j} &= \sqrt{10}{(x_{4j-3}-x_{4j})}^2 \qquad \text{ for
      $j = 1,\dots,n/4$.}
  \end{split}
\end{align}
The unique minimum $\Obj(x^*)=0$ is attained at $x^*=0$.

Problem F. The Trigonometric function, Problem (26) from
\citet{more1981testing},
\begin{align}
  \begin{split}
    \Obj(x)&={\textstyle\frac{1}{2}}\sum_{j=1}^n {t_j(x)}^2,\text{ where}\\
    t_j &= n + j(1-\cos x_j) - \sin x_j -\sum_{i=1}^n\cos(x_i).
  \end{split}
\end{align}
The unique minimum $\Obj(x^*)=0$ is attained at $x^*=0$.  Note that in
\citet{sterck2013steepest}, a minus sign is used in front of
$j(1-\cos x_j)$. We follow the original formulation of
\citet{more1981testing}.


Problem G. Penalty function I, Problem (23) from
\citet{more1981testing},
\begin{align}
  \begin{split}
    \Obj(x)&={\textstyle\frac{1}{2}}\Bigl( {t_0(x)}^2 +
    \sum_{j=1}^n {t_j(x)}^2\Bigr),\text{ where}\\
    t_0 &= -0.25 + \sum_{j=1}^nx_j^2, \text{ and}\\
    t_j &= \sqrt{10^{-5}}(x_j-1)\qquad (j=1,\dots,n).
  \end{split}
\end{align}
The minimum is not known explicitly for Problem G, and depends on the
value of $n$.

\subsection{Experiment design}\label{subsec:experiment_design}
We test the N-GMRES and O-ACCEL algorithms with two steepest descent
preconditioners
\begin{align}
  \Precon_{\text{Z}}(\Obj,x) &= x-\lambda_{\text{Z}} \frac{\Grad(x)}{\norm{\Grad(x)}_2}, \qquad
                               \text{with Z = A, B, and}\nonumber\\
  \lambda_{\text{A}} &= \text{determined by line search},\\
  \lambda_{\text{B}} &= \min(\delta,\norm{\Grad(x)}_2).
\end{align}
Thus, the two preconditioners only differ in the choice of step
length. Option A employs a globalizing strategy with a chosen line
search, whilst option B takes a predetermined step length.  By
choosing a short, predetermined step length $\delta>0$, we expand the
subspace to search for $\alpha$ and stay close to the previous iterate
$x^{(k)}$, hopefully improving the linearisations
in~\eqref{eq:grad_xA_approx} and~\eqref{eq:approx_hess}.  For the
experiments, we use the line search algorithm by \citet{more1994line},
which satisfies the Wolfe conditions \citep{nocedal2006numerical}.  It
is both employed for $\Precon_{\text{A}}$, and in the line search
$x^P+\lambda(x^A-x^P)$ between the preconditioned step $x^P$ and the
accelerated step $x^A$ of the N-GMRES and O-ACCEL routines.

To closely follow the testing conditions of
\citet{sterck2013steepest}, we use the N-CG, L-BFGS and
Mor\'{e}-Thuente line search implementations from the Poblano toolbox
by \citet{dunlavy2010poblano}.  These may not be state of the art
implementations, however, the main focus of this chapter is to
investigate the performance of the N-GMRES and O-ACCEL
algorithms. % Future work will include testing the O-ACCEL algorithm
% with appropriate preconditioners on more comprehensive test sets,
% against state of the art implementations of gradient based
% optimisation algorithms.

All optimisation procedures employ the Mor\'{e}-Thuente line search
with the following options: decrease tolerance $c_1=10^{-4}$ and
curvature tolerance $c_2=0.1$ for the Wolfe conditions, starting step
length $\lambda=1$, and a maximum of \num{20} $\Obj/\Grad$
evaluations.  The N-GMRES and O-ACCEL history lengths are set to
$w_{\textnormal{max}}=20$, and the regularisation parameter is set to
$\epsilon_0=10^{-12}$.  For $\Precon_{\text{B}}$, the fixed step
length is set to $\delta=10^{-4}$.  The L-BFGS history size is set to
\num{5}. Larger history sizes were found by \citet{sterck2013steepest}
to be harmful for the L-BFGS performance on this test set.

Note that our choice of curvature tolerance $c_2=0.1$ is different
from \citet{sterck2013steepest}, where $c_2=0.01$ was used. There are
two reasons for this.  First, our choice is often used in practice,
see \citet[Ch.~3.1]{nocedal2006numerical}, and it reduces the number
of function evaluations for all the solvers considered.  Second, we
are interested in comparing the outer solvers, however, smaller values
of $c_2$ moves work from the outer solvers to the line search
algorithms.

We test Problem A-C for both problem sizes $n=100$ and $n=200$.
Problem D is tested with
$n=\num{500},\num{1000},\num{50000},\num{100000}$.  Problem E with
$n=\num{100},\num{200},\num{50000},\num{100000}$.  Problem F is called
with $n=\num{200},\num{500}$, and finally, Problem G with
$n=\num{100},\num{200}$.  Each combination of problem and problem size
is run \num{1000} times, with the components of the initial guess
drawn uniformly random from the interval $[0,1]$. For Problem C, each
instance of the problem generates a new, random, orthogonal matrix
$Q$.  This results in \num{18000} individual tests for the comparison.
To evaluate performance, we count the number of objective evaluations
required for the algorithms to reach an iterate $x$ such that
$\Obj(x)-\Obj^* < 10^{-10}(\Obj(x^{(0)})-\Obj^*)$. A solver run is
labelled as failed if it does not reach tolerance within \num{1500}
iterations.  The minimum value $\Obj^*$ is known for Problems A-F,
however for Problem G we estimate $\Obj^*$ using the lowest value
attained across all the optimisation procedures.  The results on the
collection of \num{18000} test instances are discussed in
\Cref{subsec:perf_prof}, whilst \Cref{sec:objacc_tables} provides tables of
results on the individual problems and problem sizes.

Note that our reporting of the numerical experiments differs from that
of \citet{sterck2013steepest} in two ways: First, we run each problem
combination \num{1000} times, instead of \num{10} times. Second, we
evaluate the results based on performance profiles and tables of
quantiles, instead of solely reporting the average number of
evaluations to reach tolerance.  We believe the high number of test
runs is important for more consistent values of the statistics
reported in \Cref{sec:objacc_tables} across computers, further stabilised by using
quantiles rather than averages.

\subsection{Performance profiles}\label{subsec:perf_prof}
In order to evaluate the performance of optimisers on test sets with
problems of varying size and difficulty, \citet{dolan2002benchmarking}
proposed the use of performance profiles.  For completeness, we first
define the performance profile for our chosen metric of objective
evaluations.  Let $\mathcal{P}$ denote the test set of the
$n_p=\num{18000}$ problems, and $n_s$ the number of solvers.  For each
problem $p\in\mathcal{P}$, and solver $s$, define
\begin{equation}
  t_{p,s} = \text{number of $\Obj$ evaluations required to reach
    tolerance.}
\end{equation}
In the numerical tests we say that the solver has reached tolerance
for the problem when the relative decrease in the objective value is
at least $10^{-10}$, that is
\begin{equation}\label{eq:def_performance_measure}
  t_{p,s} = \min\{k\geq 1 \mid \Obj(x^{(k)})-\Obj^* < 10^{-10}(\Obj(x^{(0)})-\Obj^*)\}.
\end{equation}
\begin{remark}
  Note that the numbers of objective and gradient calls are the same
  for each of the optimisers considered in this chapter. This is
  due to the use of the Mor\'{e}-Thuente line search algorithm.
\end{remark}

Let $\underline{t}_p$ denote the lowest number of $\Obj$ evaluations
needed to reach tolerance for problem $p$ across all the solvers,
\begin{equation}
  \underline{t}_p=\min\{t_{p,s}\mid 1\leq s\leq n_s\}.
\end{equation}
The performance ratio measures the performance on problem $p$ by
solver $s$, as defined by
\begin{equation}
  \rho_{p,s}=t_{p,s}/\underline{t}_p.
\end{equation}
The value is bounded below by $1$, and $\rho_{p,s}=1$ for at least one
solver $s$.  If solver $s$ does not solve problem $p$, then we set
$\rho_{p,s}=\infty$.  We define the performance profile
$p_s:[1,\infty)\to[0,1]$, for solver $s$, by
\begin{equation}\label{eq:perf_profile}
  p_s(\tau) = \frac{1}{n_p}\mbox{size}\left\{p\in\mathcal{P}\mid \rho_{p,s}\leq \tau\right\}.
\end{equation}
The performance profile for a solver $s$ can be viewed as an
empirical, cumulative ``distribution'' function representing the
probability of the solver $s$ reaching tolerance within a ratio $\tau$
of the fastest solver for each problem.  In particular, $p_s(1)$ gives
the proportion of problems for which solver $s$ performed best. For
large values of $\tau$, the performance profile $p_s(\tau)$ indicates
robustness, that is, what proportion of all the test problems were
solved by the solver.

\begin{figure}[ht]
  \centering
  \includegraphics[width=0.8\textwidth,axisratio=1.7]{perf_prof_all}
  \caption[Performance profiles for Problems A--G]{Performance profiles, defined in~\eqref{eq:perf_profile},
    for Problems A--G.
    O-ACCEL preconditioned with a fixed-step steepest descent (B) and
    L-BFGS mostly
    outperform the rest, except for higher factors of $\tau$. They are
    also more robust, solving the largest proportion
    of the problems when the computational budget is large.
    % The number of data points has been downscaled when creating the plots,
    % so the flat lines provide lower bounds on the performance profiles.
  }\label{fig:perf_prof_all}
\end{figure}
\Cref{fig:perf_prof_all} plots the performance profile of the $n_s=6$
solvers considered: N-CG, L-BFGS, and N-GMRES and O-ACCEL with
steepest descent preconditioning using both a line search (A) and a
fixed step size (B).  It is clear that O-ACCEL-B and L-BFGS are the
best performers across the test set.  For \SI{44}{\percent} of the
test problems they reach tolerance in the fewest $\Obj$ evaluations,
and they also solve the largest proportion of problems within higher
factors $\tau$ of the best performance ratio.  There is also a region
where N-CG does particularly well, solving the largest proportion of
problems within two to three times the highest performing solver.  The
worst performers are N-GMRES-A and O-ACCEL-A, mainly due to the high
amount of work that the line search must do to satisfy the Wolfe
conditions along the steepest descent directions.

It is notable that O-ACCEL-B is competitive with L-BFGS on the test
set. Tests, not presented in this work, indicate that the L-BFGS
performance improves by using a line search with Wolfe curvature
condition parameter $c_2=0.9$, rather than $c_2=0.1$ as used in this
chapter.  The main focus of this chapter is, however, to
investigate the potential improvement of minimising the objective
rather than an $\ell_2$ norm of the gradient. Thus, we are more
interested in the comparison between N-GMRES and O-ACCEL.  The two
plots in \Cref{fig:perf_prof_nls_nsd} show the performance profiles
comparing N-GMRES and O-ACCEL, and in both cases show a significant
improvement by minimising the objective. In fact, O-ACCEL reaches
tolerance first on \SIrange{63}{71}{\percent} of the test problems.
The instances where N-GMRES does better is primarily in Problems E, F,
and G, as can be seen from \Cref{tbl:probEG} in \Cref{sec:objacc_tables}.  One of
the findings of \citet{sterck2013steepest} was that N-GMRES with line
search-steepest descent often stagnated or converged very slowly. From
the left plot of \Cref{fig:perf_prof_nls_nsd}, we see that this issue
is reduced with the O-ACCEL acceleration. It also turns out that
O-ACCEL-A has a larger success rate over the test set than N-GMRES-A.
\begin{figure}[hbt]
  \centering
  \begin{minipage}{0.49\textwidth}
    \includegraphics[width=\textwidth,axisratio=1.5]{perf_prof_nls}
  \end{minipage}
  \hfill
  \begin{minipage}{0.49\textwidth}
    \includegraphics[width=\textwidth,axisratio=1.5]{perf_prof_nsd}
  \end{minipage}
  \caption[Performance profiles for N-GMRES and O-ACCEL on Problems A--G]{Performance profiles comparing N-GMRES and O-ACCEL with
    steepest descent with line search (A, left) and without (B,
    right).  O-ACCEL outperforms N-GMRES in both cases on our test
    set.
    Note that the lines of O-ACCEL-A and N-GMRES-A cross in
    \Cref{fig:perf_prof_all}, but not in the left figure here, because
    the performance profiles change depending on the set of solvers
    considered.
  }\label{fig:perf_prof_nls_nsd}
\end{figure}

\subsection{The tensor optimisation problem from \citeauthor{sterck2013steepest}}\label{subsec:tensor_optim}
The original motivation for N-GMRES was to improve convergence for a
tensor optimisation problem \citep{sterck2012nonlinear}.
\citet{sterck2012nonlinear,sterck2013steepest} shows that
using N-GMRES with a domain-specific ALS preconditioner is better than
generic optimisers such as L-BFGS and N-CG.
\citet{sterck2013steepest} states that
\begin{quote}
  ``In this problem, a rank-three
  canonical tensor approximation (with 450 variables) is sought for a
  three-way data tensor of size $50\times 50\times 50$. The data tensor
  is generated starting from a canonical tensor with specified rank and
  random factor matrices that are modified to have prespecified column
  colinearity, and noise is added. This is a standard canonical tensor
  decomposition test problem \citep{acar2011scalable}.''
\end{quote}
For this
chapter, we run the \num{1000} realisations of the test problem
using the code provided by \citet{sterck2013steepest} with the
parameter values described in \Cref{subsec:experiment_design}.  The
algorithms tested for this problem are vanilla ALS, N-GMRES-ALS,
O-ACCEL-ALS, N-CG, and L-BFGS.  \Cref{fig:perf_prof_tensor_cp_all} and
\Cref{tbl:quantiles_tensor_cp_all} show the performance profiles and
quantiles for the number of $\Obj$ evaluations required to reach
tolerance.  We see that O-ACCEL-ALS and N-GMRES-ALS perform better
than the other algorithms, which underscores the advantage of applying
these acceleration methods to domain-specific algorithms.
\begin{figure}[htbp]
  \centering
  \includegraphics[width=0.8\textwidth,axisratio=1.7]{perf_prof_tensor_cp_all}
  \caption[Performance profiles from the tensor optimisation test problem]{Performance profiles from the tensor optimisation test
    problem. O-ACCEL and N-GMRES perform significantly better than the
    other solvers.}\label{fig:perf_prof_tensor_cp_all}
\end{figure}

\begin{table}[hbtp]
  \centering
  \begin{tabular}{lS[table-auto-round,table-format=4]
    S[table-auto-round,table-format=4]
    S[table-auto-round,table-format=4]}
    \toprule
    Algorithm&{$\mathcal{Q}_{0.1}$}&{$\mathcal{Q}_{0.5}$}&{$\mathcal{Q}_{0.9}$}\\
    \midrule
    ALS&877.5&1107.0&1212.0\\
    \rowcolor[gray]{0.9}
    O-ACCEL-ALS&170.0&227.0&297.0\\
    N-CG&405.0&806.5&2449.5\\
    L-BFGS&301.5&437.5&3036.5\\
    N-GMRES-ALS&169.0&234.0&311.5\\
    \bottomrule
  \end{tabular}
  \caption[Numerical results from the tensor optimisation test problem]{Numerical results from the tensor optimisation test
    problem showing statistics on number of $\Obj$ evaluations. O-ACCEL and N-GMRES perform significantly better than the
    other solvers.}\label{tbl:quantiles_tensor_cp_all}
\end{table}


\subsection{CUTEst test problems}
The test problems we have considered so far were taken from
\citet{sterck2013steepest} and originally used to promote N-GMRES. We
finish by presenting results from a numerical experiment using
problems from the CUTEst problem set \citep{gould2015cutest}.  For this
experiment, we compare the solvers O-ACCEL-B, L-BFGS, and N-GMRES-B,
with the parameter values described in
\Cref{subsec:experiment_design}.  The minima are not known for many of
the CUTEst problems, and so we change the tolerance criterion to be
defined in terms of the relative decrease of the gradient norm.  The
performance measure used for this experiment is
\begin{equation}\label{eq:def_performance_measure_cutest}
  t_{p,s} = \min\{k\geq 1 \mid \|\Grad(x^{(k)})\|_\infty \leq 10^{-8}\|\Grad(x^{(0)})\|_\infty\}.
\end{equation}
A solver run is labelled as failed if it does not reach tolerance
within \num{2000} iterations.

We run the experiment using implementations of the solvers from the
package Optim. To be sure, we have also verified
that the Optim code yields the same results as the MATLAB code for
Problems A--G.

The \num{33} problems we consider are listed in
\Cref{tbl:cutest_table1,tbl:cutest_table2} of \Cref{sec:objacc_tables} together
with the results of the numerical experiment.  We selected the
problems with dimension $n=50$ to \num{10000} that satisfy the two
criteria (i) the objective type is in the category ``other'' (ii) at
least one of the solvers succeed in reaching tolerance.
\Cref{fig:perf_prof_cutest} shows performance profiles from the
experiment. L-BFGS reaches tolerance first for most of the problems,
however, O-ACCEL reaches tolerance within \num{2000} iterations for
more of the test problems. In the problems where L-BFGS does not reach
tolerance it stops because it fails prematurely, whilst N-GMRES-B only
fails due to reaching \num{2000} iterations.  We believe the poorer
performance of the acceleration algorithms for the CUTEst problems,
compared to the previous experiments, is due to the poor performance
of the steepest descent preconditioner on these problems.  Again,
O-ACCEL-B performs better than N-GMRES-B, which underscores our claim
that accelerating based on the objective function is better than
accelerating based on the gradient norm.
\begin{figure}[htb]
  \centering
  \begin{minipage}{0.499\textwidth}
    \includegraphics[width=\textwidth,axisratio=1.5]{perf_prof_cutest_all}
  \end{minipage}%
  \begin{minipage}{0.499\textwidth}
    \includegraphics[width=\textwidth,axisratio=1.5]{perf_prof_cutest_nsd}
  \end{minipage}
  \caption[Performance profiles for the CUTEst test problems]{Performance profiles for the CUTEst test problems from
    \Cref{tbl:cutest_table1,tbl:cutest_table2}.  L-BFGS is the highest
    performing most of the time, however, O-ACCEL-B reaches tolerance
    for more problems.}\label{fig:perf_prof_cutest}
\end{figure}

\section{Discussion}\label{sec:accel_conclusion}
We have proposed a simple acceleration algorithm for optimisation,
based on the N-GMRES algorithm by
\citet{washio1997krylov,sterck2013steepest}.  N-GMRES for optimisation
aims to accelerate a solver step when solving the nonlinear system
$\nabla\Obj(x)=0$ by minimising the residual in the $\ell_2$ norm over
a subspace from previous iterates.  The acceleration step consists of
solving a small linear system that arises from a linearisation of the
gradient.

We propose to take advantage of the structure of the optimisation
problem and instead accelerate based on the objective value
$\Obj(x)$.  This new approach, labelled O-ACCEL, shows a significant
improvement to the original N-GMRES algorithm in numerical tests when
accelerating a steepest descent solver. The first test problems are
taken from \citet{sterck2013steepest} and run under the same
conditions that proved to be beneficial for N-GMRES.  Further tests on
a selection of CUTEst problems strengthen the conclusion that O-ACCEL
outperforms N-GMRES.  Another strength of these acceleration
algorithms is that they can be combined with many types of
optimisers. We have seen O-ACCEL's efficiency with steepest descent,
and accelerating quasi-Newton, Newton methods, and domain-specific
methods have potential to reduce costs for more expensive algorithms.
For example, in \citet{sterck2013steepest} it is shown that N-GMRES
significantly accelerates the alternating least squares algorithm
(ALS), which already without acceleration performs much better than
L-BFGS and N-CG on a standard canonical tensor decomposition
problem. Our numerical tests show that O-ACCEL further improves the
ALS convergence for this problem.

There are two particular paths of interest to improve the proposed
acceleration scheme. The first is to reduce the cost by not using a
line search between the proposed steps by the solver and O-ACCEL. One
can instead rely on heuristics along the lines of those proposed by
\citet{washio1997krylov}.  The second is to find better heuristics for
choosing previous iterates to use in the acceleration step. Currently,
no choices are made, other than discarding all iterates when problems
appear.  Better guidelines for the number of previous iterates to
store is another topic of interest, especially when memory storage is
limited.

We would like to investigate connections between the proposed O-ACCEL
acceleration step and other optimisation procedures, in the same
fashion that \citet{fang2009two} put Anderson acceleration in the
context of a family of Broyden-type approximations of the inverse
Jacobian (Hessian).  The preliminary analysis presented in this
chapter shows that, for convex quadratic objectives, O-ACCEL with a
gradient descent preconditioner is equivalent to FOM for linear
systems. As FOM is equivalent to CG for symmetric positive definite
systems, we can view O-ACCEL in the context of N-CG methods using a
larger history size than usual.  There are many new ideas for
improving step directions based on previous iterates, such as the
acceleration scheme by \citet{damien2016regularized}, and Block BFGS
by \citet{gao2016block}.  A better understanding of the overlaps
between these and more classical optimisation procedures can provide
useful guidance for further research.

Further work is needed to test O-ACCEL on a wider range of problems,
with comparisons to other state-of-the-art implementations of solvers
and accelerators, in order to provide guidance as to when a method is
appropriate.  For example, on Problems A--G, O-ACCEL accelerating
steepest descent is superior to N-CG and slightly better than
L-BFGS. These results may, however, be due to implementations from
\citet{sterck2013steepest} and test problems favouring the
acceleration algorithms. They are still indicative of the power of
objective value based optimisation, a research track that is worth
pursuing further.


\paragraph*{Data access.}
The MATLAB code used in producing this chapter is available at
\url{https://github.com/anriseth/objective_accel_code}.  It includes
implementations of the N-GMRES and O-ACCEL algorithms, and code
to run Problems A--G as well as the tensor optimisation problem.

\section{Tables of numerical results}\label{sec:objacc_tables}
To supplement the performance profiles in the chapter, we include
tables that present statistics of the solver performances for the
individual test problems.

\Cref{tbl:probAC,tbl:probD,tbl:probEG} show the results from test problems A--G.
Each of the problems were tested with different sizes $n$, and for
each value $n$ the problems were run \num{1000} times in order to
create statistics.  The tables report the \num{0.1}, \num{0.5}, and
\num{0.9} quantiles of $\Obj$ evaluations to reach the objective value
reduction in~\eqref{eq:def_performance_measure}, denoted by
$\mathcal{Q}_{0.1}$, $\mathcal{Q}_{0.5}$, and $\mathcal{Q}_{0.9}$
respectively.  \Cref{tbl:probAC} provides results for Problems A--C,
\Cref{tbl:probD} for Problem D,
and \Cref{tbl:probEG} for the remaining Problems E--G.

\Cref{tbl:cutest_table1,tbl:cutest_table2} show the results from the
CUTEst problems, where ``Fail'' means failure to reach the gradient
value reduction in~\eqref{eq:def_performance_measure_cutest} within
\num{2000} iterations.  The norm used for the gradient values in the
tables is the infinity norm.

\begin{table}[hp]
  \sisetup{detect-weight=true,detect-inline-weight=math}
  \centering
  \begin{tabular}{l*{2}{
    S[table-auto-round,table-format=4]
    S[table-auto-round,table-format=4]
    S[table-auto-round,table-format=4]}
    }
    \toprule
    Algorithm
    &\multicolumn{3}{c}{A, $n=100$}
    &\multicolumn{3}{c}{A, $n=200$}\\
    \cmidrule(lr){2-4}\cmidrule(lr){5-7}
    &{$\mathcal{Q}_{0.1}$}&{$\mathcal{Q}_{0.5}$}&{$\mathcal{Q}_{0.9}$}
                          &{$\mathcal{Q}_{0.1}$}&{$\mathcal{Q}_{0.5}$}&{$\mathcal{Q}_{0.9}$}\\
    \midrule
    \rowcolor[gray]{0.9}
    O-ACCEL-B&75.0&79.0&81.0&103.0&107.0&111.0\\
    O-ACCEL-A&131.0&136.0&140.0&171.0&179.0&184.0\\
    N-CG&87.0&93.0&99.0&113.0&131.0&145.0\\
    \rowcolor[gray]{0.9}
    L-BFGS&75.0&79.0&81.0&103.0&107.0&111.0\\
    N-GMRES-B&111.0&117.0&122.0&158.0&169.0&192.0\\
    N-GMRES-A&166.0&246.0&335.5&306.5&414.0&510.0\\
    \bottomrule
  \end{tabular}
  \\[1em]
  \begin{tabular}{l*{2}{
    S[table-auto-round,table-format=4]
    S[table-auto-round,table-format=4]
    S[table-auto-round,table-format=4]}
    }
    \toprule
    Algorithm
    &\multicolumn{3}{c}{B, $n=100$}
    &\multicolumn{3}{c}{B, $n=200$}\\
    \cmidrule(lr){2-4}\cmidrule(lr){5-7}
    &{$\mathcal{Q}_{0.1}$}&{$\mathcal{Q}_{0.5}$}&{$\mathcal{Q}_{0.9}$}
                          &{$\mathcal{Q}_{0.1}$}&{$\mathcal{Q}_{0.5}$}&{$\mathcal{Q}_{0.9}$}\\
    \midrule
    O-ACCEL-B&183.0&267.0&415.5&262.0&364.5&595.0\\
    O-ACCEL-A&258.0&389.0&545.5&377.0&478.0&799.5\\
    N-CG&134.0&211.0&560.0&221.0&359.0&1598.0\\
    \rowcolor[gray]{0.9}
    L-BFGS&76.0&100.0&169.0&99.0&127.0&292.0\\
    N-GMRES-B&215.0&314.5&541.5&317.0&433.0&839.5\\
    N-GMRES-A&272.0&648.0&1515.5&452.0&809.0&2203.5\\
    \bottomrule
  \end{tabular}
  \\[1em]
  \begin{tabular}{l*{2}{
    S[table-auto-round,table-format=4]
    S[table-auto-round,table-format=4]
    S[table-auto-round,table-format=4]}
    }
    \toprule
    Algorithm
    &\multicolumn{3}{c}{C, $n=100$}
    &\multicolumn{3}{c}{C, $n=200$}\\
    \cmidrule(lr){2-4}\cmidrule(lr){5-7}
    &{$\mathcal{Q}_{0.1}$}&{$\mathcal{Q}_{0.5}$}&{$\mathcal{Q}_{0.9}$}
                          &{$\mathcal{Q}_{0.1}$}&{$\mathcal{Q}_{0.5}$}&{$\mathcal{Q}_{0.9}$}\\
    \midrule
    O-ACCEL-B&112.5&136.0&177.5&151.0&176.0&214.5\\
    O-ACCEL-A&187.5&208.0&258.5&264.0&292.0&324.0\\
    N-CG&165.0&187.0&215.0&259.0&298.0&344.0\\
    \rowcolor[gray]{0.9}
    L-BFGS&104.0&114.0&125.0&147.5&160.0&177.0\\
    N-GMRES-B&142.0&164.0&208.0&219.0&253.5&304.0\\
    N-GMRES-A&264.0&333.0&459.0&508.0&620.0&854.0\\
    \bottomrule
  \end{tabular}
  \caption[Quantiles reporting $\Obj$ evaluations to reach
    tolerance for in Problems A--C]{Quantiles reporting $\Obj$ evaluations to reach
    tolerance for each solver. Grey rows highlight the solver with the
    best \num{0.5} qauntile.
    L-BFGS performs best for these easier problems.  In Problem A, the L-BFGS and
    O-ACCEL performance measures are so similar that the quantiles are
    the same.}\label{tbl:probAC}
\end{table}

\begin{table}[p]
  \sisetup{detect-weight=true,detect-inline-weight=math}
  \centering
  \begin{tabular}{l*{2}{
    S[table-auto-round,table-format=4]
    S[table-auto-round,table-format=4]
    S[table-auto-round,table-format=4]}
    }
    \toprule
    Algorithm
    &\multicolumn{3}{c}{D, $n=500$}
    &\multicolumn{3}{c}{D, $n=1000$}\\
    \cmidrule(lr){2-4}\cmidrule(lr){5-7}
    &{$\mathcal{Q}_{0.1}$}&{$\mathcal{Q}_{0.5}$}&{$\mathcal{Q}_{0.9}$}
                          &{$\mathcal{Q}_{0.1}$}&{$\mathcal{Q}_{0.5}$}&{$\mathcal{Q}_{0.9}$}\\
    \midrule
    \rowcolor[gray]{0.9}
    O-ACCEL-B&93.0&105.0&123.0&91.0&98.0&116.0\\
    O-ACCEL-A&193.0&233.0&276.5&192.0&233.0&280.0\\
    N-CG&158.0&188.0&196.0&162.0&190.0&197.0\\
    L-BFGS&128.0&155.0&194.0&128.5&153.0&188.5\\
    N-GMRES-B&141.0&163.0&193.0&142.0&167.0&193.0\\
    N-GMRES-A&284.0&349.0&508.0&290.0&349.0&470.5\\
    \midrule
    Algorithm
    &\multicolumn{3}{c}{D, $n=50000$}
    &\multicolumn{3}{c}{D, $n=100000$}\\
    \cmidrule(lr){2-4}\cmidrule(lr){5-7}
    &{$\mathcal{Q}_{0.1}$}&{$\mathcal{Q}_{0.5}$}&{$\mathcal{Q}_{0.9}$}
                          &{$\mathcal{Q}_{0.1}$}&{$\mathcal{Q}_{0.5}$}&{$\mathcal{Q}_{0.9}$}\\
    \midrule
    \rowcolor[gray]{0.9}
    O-ACCEL-B&101.0&117.0&132.0&122.0&126.0&135.0\\
    O-ACCEL-A&195.0&226.0&276.0&196.0&225.0&271.0\\
    N-CG&159.0&187.0&196.0&159.0&188.0&196.0\\
    L-BFGS&131.0&156.0&190.0&130.0&156.0&191.0\\
    N-GMRES-B&154.0&178.0&215.0&162.5&190.0&230.5\\
    N-GMRES-A&312.0&378.0&525.0&319.5&394.0&561.5\\
    \bottomrule
  \end{tabular}
  \caption[Quantiles reporting $\Obj$ evaluations to reach
    tolerance in Problem D]{Quantiles reporting $\Obj$ evaluations to reach
    tolerance for each solver. Grey rows highlight the solver with the
    best \num{0.5} qauntile.
    O-ACCEL-B handles Problem D best.}\label{tbl:probD}
\end{table}

\begin{table}[p]
  \centering
  \begin{tabular}{l*{2}{
    S[table-auto-round,table-format=4]
    S[table-auto-round,table-format=4]
    S[table-auto-round,table-format=4]}
    }
    \toprule
    Algorithm
    &\multicolumn{3}{c}{E, $n=100$}
    &\multicolumn{3}{c}{E, $n=200$}\\
    \cmidrule(lr){2-4}\cmidrule(lr){5-7}
    &{$\mathcal{Q}_{0.1}$}&{$\mathcal{Q}_{0.5}$}&{$\mathcal{Q}_{0.9}$}
                          &{$\mathcal{Q}_{0.1}$}&{$\mathcal{Q}_{0.5}$}&{$\mathcal{Q}_{0.9}$}\\
    \midrule
    \rowcolor[gray]{0.9}
    O-ACCEL-B&190.0&222.0&265.0&198.0&228.0&273.5\\
    O-ACCEL-A&301.0&349.0&624.5&312.0&371.0&780.5\\
    N-CG&204.5&238.0&283.0&213.0&245.0&290.0\\
    L-BFGS&463.0&626.5&964.5&479.5&638.5&1035.5\\
    N-GMRES-B&231.5&267.0&330.0&235.0&268.0&337.5\\
    N-GMRES-A&280.0&332.0&395.0&284.0&335.0&401.0\\
    \midrule
    Algorithm
    &\multicolumn{3}{c}{E, $n=50000$}
    &\multicolumn{3}{c}{E, $n=100000$}\\
    \cmidrule(lr){2-4}\cmidrule(lr){5-7}
    &{$\mathcal{Q}_{0.1}$}&{$\mathcal{Q}_{0.5}$}&{$\mathcal{Q}_{0.9}$}
                          &{$\mathcal{Q}_{0.1}$}&{$\mathcal{Q}_{0.5}$}&{$\mathcal{Q}_{0.9}$}\\
    \midrule
    O-ACCEL-B&368.0&487.0&689.0&399.5&536.0&798.0\\
    O-ACCEL-A&744.5&1157.0&1355.5&763.5&1207.0&1417.0\\
    N-CG&297.0&360.0&461.0&321.0&384.0&490.5\\
    L-BFGS&599.0&703.0&851.5&625.5&725.0&879.0\\
    \rowcolor[gray]{0.9}
    N-GMRES-B&275.0&335.0&738.0&258.0&318.0&848.0\\
    N-GMRES-A&309.5&390.5&561.5&318.0&402.0&609.0\\
    \bottomrule
  \end{tabular}
  \\[1em]
  \begin{tabular}{l*{2}{
    S[table-auto-round,table-format=4]
    S[table-auto-round,table-format=4]
    S[table-auto-round,table-format=4]}
    }
    \toprule
    Algorithm
    &\multicolumn{3}{c}{F, $n=200$}
    &\multicolumn{3}{c}{F, $n=500$}\\
    \cmidrule(lr){2-4}\cmidrule(lr){5-7}
    &{$\mathcal{Q}_{0.1}$}&{$\mathcal{Q}_{0.5}$}&{$\mathcal{Q}_{0.9}$}
                          &{$\mathcal{Q}_{0.1}$}&{$\mathcal{Q}_{0.5}$}&{$\mathcal{Q}_{0.9}$}\\
    \midrule
    O-ACCEL-B&53.0&71.0&118.0&44.0&55.0&96.5\\
    O-ACCEL-A&81.0&93.0&110.0&84.0&102.0&121.0\\
    \rowcolor[gray]{0.9}
    N-CG&34.0&46.0&60.0&33.0&47.0&69.0\\
    \rowcolor[gray]{0.9}
    L-BFGS&41.0&48.0&56.0&34.0&44.0&51.0\\
    N-GMRES-B&48.0&59.0&110.0&43.0&51.0&88.5\\
    N-GMRES-A&76.0&87.0&99.0&78.0&92.0&107.0\\
    \bottomrule
  \end{tabular}
  \\[1em]
  \begin{tabular}{l*{2}{
    S[table-auto-round,table-format=4]
    S[table-auto-round,table-format=4]
    S[table-auto-round,table-format=4]}
    }
    \toprule
    Algorithm
    &\multicolumn{3}{c}{G, $n=100$}
    &\multicolumn{3}{c}{G, $n=200$}\\
    \cmidrule(lr){2-4}\cmidrule(lr){5-7}
    &{$\mathcal{Q}_{0.1}$}&{$\mathcal{Q}_{0.5}$}&{$\mathcal{Q}_{0.9}$}
                          &{$\mathcal{Q}_{0.1}$}&{$\mathcal{Q}_{0.5}$}&{$\mathcal{Q}_{0.9}$}\\
    \midrule
    O-ACCEL-B&148.0&211.5&296.0&195.5&224.0&257.5\\
    O-ACCEL-A&301.5&940.0&1078.0&220.0&815.0&956.5\\
    N-CG&76.0&191.0&201.0&53.0&165.0&174.0\\
    \rowcolor[gray]{0.9}
    L-BFGS&66.0&173.0&180.0&53.0&150.0&156.0\\
    N-GMRES-B&161.0&216.0&266.0&166.5&210.0&245.0\\
    N-GMRES-A&528.0&764.0&4518.0&203.0&720.0&4526.0\\
    \bottomrule
  \end{tabular}
  \caption[Quantiles reporting $\Obj$ evaluations to reach
    tolerance in Problems E--G]{Quantiles reporting $\Obj$ evaluations to reach
    tolerance for each solver. Grey rows highlight the solver with the
    best \num{0.5} qauntile.
    N-GMRES performs best at the median range for two problems,
    however it is less robust as can be seen from the upper quantile.
  }\label{tbl:probEG}
\end{table}

\begin{table}[p]
  \scriptsize
  \centering
  \ifsubmission
  \sisetup{tight-spacing=true,table-auto-round}
\begin{tabular}{*{3}{l}*{2}{S[table-format=4]}
*{2}{S[table-format=-1.1e-2]}c}
  \toprule
  \multicolumn{2}{c}{Problem}
  & Solver & {Iter} & {$\Obj$-calls} & {$\Obj_{\textnormal{min}}$} &
  {$\|\Grad_{\textnormal{min}}\|$} & Fail \\
  \midrule
\multirow{3}{*}{ARWHEAD} & $ n = \num{5000} $  & O-ACCEL-B & 9 & 20 & 0.00e+00 & 2.33e-06 &  \\
 & $\Obj_0 = \num{1.5e+04} $  & L-BFGS & 6 & 21 & 0.00e+00 & 4.82e-07 &  \\
 & ${\|\Grad_0\|} = \num{4.0e+04} $  & N-GMRES-B & 6 & 20 & 2.63e-09 & 2.14e-05 &  \\
\addlinespace
\multirow{3}{*}{BOX} & $ n = \num{10000} $  & O-ACCEL-B & 19 & 86 & -1.86e+03 & 1.95e-10 &  \\
 & $\Obj_0 = \num{0.0e+00} $  & L-BFGS & 7 & 14 & -1.86e+03 & 4.00e-09 &  \\
 & ${\|\Grad_0\|} = \num{5.0e-01} $  & N-GMRES-B & 30 & 105 & -1.86e+03 & 2.19e-09 &  \\
\addlinespace
\multirow{3}{*}{COSINE} & $ n = \num{10000} $  & O-ACCEL-B & 52 & 272 & -1.00e+04 & 7.14e-09 &  \\
 & $\Obj_0 = \num{8.8e+03} $  & L-BFGS & 12 & 22 & -1.00e+04 & 3.10e-09 &  \\
 & ${\|\Grad_0\|} = \num{9.6e-01} $  & N-GMRES-B & 29 & 80 & -1.00e+04 & 1.96e-09 &  \\
\addlinespace
\multirow{3}{*}{CRAGGLVY} & $ n = \num{5000} $  & O-ACCEL-B & 157 & 478 & 1.69e+03 & 4.71e-05 &  \\
 & $\Obj_0 = \num{2.7e+06} $  & L-BFGS & 57 & 133 & 1.69e+03 & 4.01e-05 &  \\
 & ${\|\Grad_0\|} = \num{5.6e+03} $  & N-GMRES-B & 229 & 760 & 1.69e+03 & 3.09e-05 &  \\
\addlinespace
\multirow{3}{*}{DIXMAANA} & $ n = \num{3000} $  & O-ACCEL-B & 48 & 223 & 1.00e+00 & 6.11e-10 &  \\
 & $\Obj_0 = \num{2.9e+04} $  & L-BFGS & 6 & 12 & 1.00e+00 & 9.90e-16 &  \\
 & ${\|\Grad_0\|} = \num{2.8e+01} $  & N-GMRES-B & 13 & 53 & 1.00e+00 & 1.47e-10 &  \\
\addlinespace
\multirow{3}{*}{DIXMAANB} & $ n = \num{3000} $  & O-ACCEL-B & 31 & 99 & 1.00e+00 & 3.16e-08 &  \\
 & $\Obj_0 = \num{4.7e+04} $  & L-BFGS & 6 & 11 & 1.00e+00 & 2.25e-07 &  \\
 & ${\|\Grad_0\|} = \num{4.0e+01} $  & N-GMRES-B & 35 & 228 & 1.00e+00 & 3.79e-10 &  \\
\addlinespace
\multirow{3}{*}{DIXMAANC} & $ n = \num{3000} $  & O-ACCEL-B & 30 & 105 & 1.00e+00 & 4.05e-08 &  \\
 & $\Obj_0 = \num{8.2e+04} $  & L-BFGS & 7 & 14 & 1.00e+00 & 2.27e-08 &  \\
 & ${\|\Grad_0\|} = \num{7.6e+01} $  & N-GMRES-B & 13 & 49 & 1.00e+00 & 4.40e-08 &  \\
\addlinespace
\multirow{3}{*}{DIXMAAND} & $ n = \num{3000} $  & O-ACCEL-B & 23 & 75 & 1.00e+00 & 1.47e-06 &  \\
 & $\Obj_0 = \num{1.6e+05} $  & L-BFGS & 8 & 16 & 1.00e+00 & 2.55e-07 &  \\
 & ${\|\Grad_0\|} = \num{1.5e+02} $  & N-GMRES-B & 16 & 78 & 1.00e+00 & 4.89e-07 &  \\
\addlinespace
\multirow{3}{*}{DIXMAANE} & $ n = \num{3000} $  & O-ACCEL-B & 394 & 1109 & 1.00e+00 & 2.61e-07 &  \\
 & $\Obj_0 = \num{2.2e+04} $  & L-BFGS & 241 & 485 & 1.00e+00 & 2.67e-07 &  \\
 & ${\|\Grad_0\|} = \num{2.7e+01} $  & N-GMRES-B & 885 & 2751 & 1.00e+00 & 2.60e-07 &  \\
\addlinespace
\multirow{3}{*}{DIXMAANF} & $ n = \num{3000} $  & O-ACCEL-B & 282 & 765 & 1.00e+00 & 3.47e-07 &  \\
 & $\Obj_0 = \num{4.1e+04} $  & L-BFGS & 195 & 393 & 1.00e+00 & 3.73e-07 &  \\
 & ${\|\Grad_0\|} = \num{3.9e+01} $  & N-GMRES-B & 567 & 1687 & 1.00e+00 & 3.87e-07 &  \\
\addlinespace
\multirow{3}{*}{DIXMAANG} & $ n = \num{3000} $  & O-ACCEL-B & 277 & 770 & 1.00e+00 & 6.93e-07 &  \\
 & $\Obj_0 = \num{7.6e+04} $  & L-BFGS & 165 & 334 & 1.00e+00 & 6.88e-07 &  \\
 & ${\|\Grad_0\|} = \num{7.5e+01} $  & N-GMRES-B & 570 & 1673 & 1.00e+00 & 7.16e-07 &  \\
\addlinespace
\multirow{3}{*}{DIXMAANH} & $ n = \num{3000} $  & O-ACCEL-B & 236 & 655 & 1.00e+00 & 1.51e-06 &  \\
 & $\Obj_0 = \num{1.5e+05} $  & L-BFGS & 146 & 297 & 1.00e+00 & 1.49e-06 &  \\
 & ${\|\Grad_0\|} = \num{1.5e+02} $  & N-GMRES-B & 620 & 1835 & 1.00e+00 & 1.38e-06 &  \\
\addlinespace
\multirow{3}{*}{DIXMAANK} & $ n = \num{3000} $  & O-ACCEL-B & 862 & 1877 & 1.00e+00 & 7.09e-07 &  \\
 & $\Obj_0 = \num{7.4e+04} $  & L-BFGS & 392 & 787 & 1.00e+00 & 7.06e-07 &  \\
 & ${\|\Grad_0\|} = \num{7.4e+01} $  & N-GMRES-B & 556 & 1640 & 1.00e+00 & 7.17e-07 &  \\
\addlinespace
\multirow{3}{*}{DIXMAANL} & $ n = \num{3000} $  & O-ACCEL-B & 267 & 685 & 1.00e+00 & 1.44e-06 &  \\
 & $\Obj_0 = \num{1.5e+05} $  & L-BFGS & 240 & 485 & 1.00e+00 & 1.47e-06 &  \\
 & ${\|\Grad_0\|} = \num{1.5e+02} $  & N-GMRES-B & 378 & 1096 & 1.00e+00 & 1.35e-06 &  \\
\addlinespace
\multirow{3}{*}{DIXMAANP} & $ n = \num{3000} $  & O-ACCEL-B & 1426 & 3149 & 1.00e+00 & 1.25e-06 &  \\
 & $\Obj_0 = \num{7.1e+04} $  & L-BFGS & 549 & 1102 & 1.00e+00 & 1.25e-06 &  \\
 & ${\|\Grad_0\|} = \num{1.3e+02} $  & N-GMRES-B & 1037 & 3091 & 1.00e+00 & 1.11e-06 &  \\
\addlinespace
\multirow{3}{*}{EDENSCH} & $ n = \num{2000} $  & O-ACCEL-B & 75 & 246 & 1.20e+04 & 1.83e-05 &  \\
 & $\Obj_0 = \num{7.4e+06} $  & L-BFGS & 21 & 45 & 1.20e+04 & 1.96e-05 &  \\
 & ${\|\Grad_0\|} = \num{2.2e+03} $  & N-GMRES-B & 54 & 200 & 1.20e+04 & 7.11e-06 &  \\
\bottomrule
\end{tabular}

  \else
  \todo[inline]{Uncomment table inclusion}
  \fi
  \caption{Results from the CUTEst problems.}\label{tbl:cutest_table1}
\end{table}

\begin{table}[p]
  \scriptsize
  \centering
  \ifsubmission
  \sisetup{tight-spacing=true,table-auto-round}
\begin{tabular}{*{3}{l}*{2}{S[table-format=4]}
*{2}{S[table-format=-1.1e-2]}c}
  \toprule
  \multicolumn{2}{c}{Problem}
  & Solver & {Iter} & {$\Obj$-calls} & {$\Obj_{\textnormal{min}}$} &
  {$\|\Grad_{\textnormal{min}}\|$} & Fail \\
  \midrule
\multirow{3}{*}{EG2} & $ n = \num{1000} $  & O-ACCEL-B & 6 & 14 & -9.99e+02 & 1.68e-06 &  \\
 & $\Obj_0 = \num{-8.4e+02} $  & L-BFGS & 3 & 9 & -9.99e+02 & 4.06e-07 &  \\
 & ${\|\Grad_0\|} = \num{5.4e+02} $  & N-GMRES-B & 6 & 14 & -9.99e+02 & 3.64e-06 &  \\
\addlinespace
\multirow{3}{*}{ENGVAL1} & $ n = \num{5000} $  & O-ACCEL-B & 47 & 250 & 5.55e+03 & 8.52e-07 &  \\
 & $\Obj_0 = \num{2.9e+05} $  & L-BFGS & 33 & 366 & 5.55e+03 & 2.75e-06 & $\times$ \\
 & ${\|\Grad_0\|} = \num{1.2e+02} $  & N-GMRES-B & 65 & 318 & 5.55e+03 & 7.60e-07 &  \\
\addlinespace
\multirow{3}{*}{FLETBV3M} & $ n = \num{5000} $  & O-ACCEL-B & 143 & 989 & -2.48e+05 & 4.83e-09 &  \\
 & $\Obj_0 = \num{2.0e+02} $  & L-BFGS & 24 & 62 & -2.49e+05 & 5.12e-10 &  \\
 & ${\|\Grad_0\|} = \num{7.1e-01} $  & N-GMRES-B & 92 & 756 & -2.49e+05 & 5.02e-09 &  \\
\addlinespace
\multirow{3}{*}{FMINSRF2} & $ n = \num{5625} $  & O-ACCEL-B & 1510 & 4384 & 1.00e+00 & 9.58e-09 & $\times$ \\
 & $\Obj_0 = \num{2.8e+01} $  & L-BFGS & 1537 & 3367 & 1.00e+00 & 2.23e-10 &  \\
 & ${\|\Grad_0\|} = \num{2.4e-02} $  & N-GMRES-B & 2000 & 4661 & 1.49e+00 & 9.13e-03 & $\times$ \\
\addlinespace
\multirow{3}{*}{FMINSURF} & $ n = \num{5625} $  & O-ACCEL-B & 1741 & 5007 & 1.00e+00 & 9.72e-09 & $\times$ \\
 & $\Obj_0 = \num{2.9e+01} $  & L-BFGS & 781 & 1672 & 1.00e+00 & 2.21e-10 &  \\
 & ${\|\Grad_0\|} = \num{2.3e-02} $  & N-GMRES-B & 2000 & 4440 & 1.64e+00 & 9.19e-03 & $\times$ \\
\addlinespace
\multirow{3}{*}{NCB20} & $ n = \num{5010} $  & O-ACCEL-B & 526 & 1610 & -1.13e+03 & 3.11e-08 &  \\
 & $\Obj_0 = \num{1.0e+04} $  & L-BFGS & 467 & 1089 & -1.15e+03 & 1.29e-06 & $\times$ \\
 & ${\|\Grad_0\|} = \num{4.0e+00} $  & N-GMRES-B & 2000 & 4093 & -1.14e+03 & 6.81e-01 & $\times$ \\
\addlinespace
\multirow{3}{*}{NCB20B} & $ n = \num{5000} $  & O-ACCEL-B & 1178 & 4121 & 7.35e+03 & 3.25e-08 &  \\
 & $\Obj_0 = \num{1.0e+04} $  & L-BFGS & 1467 & 4173 & 7.35e+03 & 2.97e-05 & $\times$ \\
 & ${\|\Grad_0\|} = \num{4.0e+00} $  & N-GMRES-B & 2000 & 3013 & 7.35e+03 & 2.72e-05 & $\times$ \\
\addlinespace
\multirow{3}{*}{NONDQUAR} & $ n = \num{5000} $  & O-ACCEL-B & 363 & 1044 & 1.86e-04 & 1.67e-04 &  \\
 & $\Obj_0 = \num{5.0e+03} $  & L-BFGS & 208 & 436 & 9.30e-05 & 1.81e-04 &  \\
 & ${\|\Grad_0\|} = \num{2.0e+04} $  & N-GMRES-B & 480 & 1394 & 3.77e-04 & 1.71e-04 &  \\
\addlinespace
\multirow{3}{*}{PENALTY3} & $ n = \num{200} $  & O-ACCEL-B & 283 & 1798 & 1.00e-03 & 1.50e-03 &  \\
 & $\Obj_0 = \num{1.6e+09} $  & L-BFGS & 3 & 15 & 1.58e+09 & 1.59e+05 & $\times$ \\
 & ${\|\Grad_0\|} = \num{1.6e+05} $  & N-GMRES-B & 2000 & 2165 & 2.37e+172 & 2.37e+172 & $\times$ \\
\addlinespace
\multirow{3}{*}{POWELLSG} & $ n = \num{5000} $  & O-ACCEL-B & 115 & 460 & 3.31e-06 & 1.22e-06 &  \\
 & $\Obj_0 = \num{2.7e+05} $  & L-BFGS & 20 & 49 & 4.23e-10 & 5.52e-07 &  \\
 & ${\|\Grad_0\|} = \num{3.1e+02} $  & N-GMRES-B & 105 & 308 & 3.16e-09 & 1.41e-08 &  \\
\addlinespace
\multirow{3}{*}{POWER} & $ n = \num{10000} $  & O-ACCEL-B & 186 & 637 & 4.03e+04 & 1.52e+04 &  \\
 & $\Obj_0 = \num{2.5e+15} $  & L-BFGS & 38 & 107 & 4.93e+04 & 1.48e+04 &  \\
 & ${\|\Grad_0\|} = \num{2.0e+12} $  & N-GMRES-B & 607 & 1868 & 7.46e+04 & 1.35e+04 &  \\
\addlinespace
\multirow{3}{*}{SCHMVETT} & $ n = \num{5000} $  & O-ACCEL-B & 74 & 208 & -1.50e+04 & 1.02e-08 &  \\
 & $\Obj_0 = \num{-1.4e+04} $  & L-BFGS & 55 & 133 & -1.50e+04 & 9.19e-09 &  \\
 & ${\|\Grad_0\|} = \num{1.1e+00} $  & N-GMRES-B & 78 & 239 & -1.50e+04 & 9.29e-09 &  \\
\addlinespace
\multirow{3}{*}{SINQUAD} & $ n = \num{5000} $  & O-ACCEL-B & 78 & 297 & -6.75e+06 & 4.94e-05 &  \\
 & $\Obj_0 = \num{6.6e-01} $  & L-BFGS & 12 & 45 & -6.76e+06 & 1.38e-05 &  \\
 & ${\|\Grad_0\|} = \num{5.0e+03} $  & N-GMRES-B & 95 & 483 & -6.75e+06 & 7.10e-06 &  \\
\addlinespace
\multirow{3}{*}{SPARSQUR} & $ n = \num{10000} $  & O-ACCEL-B & 134 & 598 & 1.69e-07 & 4.35e-06 &  \\
 & $\Obj_0 = \num{1.4e+07} $  & L-BFGS & 26 & 91 & 1.36e-06 & 2.25e-04 &  \\
 & ${\|\Grad_0\|} = \num{3.2e+04} $  & N-GMRES-B & 214 & 797 & 1.08e-06 & 1.82e-05 &  \\
\addlinespace
\multirow{3}{*}{TOINTGOR} & $ n = \num{50} $  & O-ACCEL-B & 201 & 538 & 1.37e+03 & 1.34e-06 &  \\
 & $\Obj_0 = \num{5.1e+03} $  & L-BFGS & 126 & 265 & 1.37e+03 & 9.81e-07 &  \\
 & ${\|\Grad_0\|} = \num{1.5e+02} $  & N-GMRES-B & 287 & 795 & 1.37e+03 & 1.29e-06 &  \\
\addlinespace
\multirow{3}{*}{TOINTPSP} & $ n = \num{50} $  & O-ACCEL-B & 277 & 963 & 2.26e+02 & 6.29e-08 &  \\
 & $\Obj_0 = \num{1.8e+03} $  & L-BFGS & 127 & 334 & 2.26e+02 & 2.89e-07 &  \\
 & ${\|\Grad_0\|} = \num{3.0e+01} $  & N-GMRES-B & 423 & 1287 & 2.26e+02 & 2.51e-07 &  \\
\addlinespace
\multirow{3}{*}{VARDIM} & $ n = \num{200} $  & O-ACCEL-B & 8 & 28 & 6.86e-02 & 1.15e+02 &  \\
 & $\Obj_0 = \num{3.3e+16} $  & L-BFGS & 1 & 20 & 3.26e+16 & 1.94e+15 & $\times$ \\
 & ${\|\Grad_0\|} = \num{1.9e+15} $  & N-GMRES-B & 4 & 39 & 5.33e+03 & 4.98e+05 &  \\
\bottomrule
\end{tabular}

  \else
  \todo[inline]{Uncomment table inclusion}
  \fi
  \caption{Results from the CUTEst tests.}\label{tbl:cutest_table2}
\end{table}



\biblio{} % Bibliography when standalone
\end{document}

%%% Local Variables:
%%% mode: latex
%%% TeX-master: t
%%% TeX-command-extra-options: "--shell-escape"
%%% End:
