\documentclass[main.tex]{subfiles}


\begin{document}

\chapter{Continuous-time pricing with diffusion
  models}\label{ch:cts_control}

In the previous two chapters we considered decisions with no temporal
structure and with a discrete, sequential temporal structure.
As the frequency of sequential decisions increases there may be an
advantage in using a continuum approximation.
We consider a continuous-time version of the pricing problem from
\Cref{ch:discrete_control} to motivate such decision problems.
% The work which this chapter is based on has been submitted to
% the ``IMA Journal of Management Mathematics''.\todo{Update with publication}

Consider a monopolist retailer who wants to design a dynamic pricing policy
for a product over a given period, in order to maximise their total
revenue and minimise the cost associated with handling unsold items
at a given terminal time.
Two important components in the decision process are the ability to
take into account the uncertainty associated with future cost and demand and to
optimally adjust for new knowledge as it arrives.
We illustrate how to address both components in this chapter,
focusing on large-inventory limits and multiplicative demand
uncertainty.
Assume the retailer sells large volumes of its product at
high frequency. In such a setting, it is
appropriate to model the sales process in a continuum limit, both for
product volume and for time, similar to \citet{kalish1983monopolist}.

In the revenue management literature, most efforts to model demand uncertainty in
continuous time have focused on discrete Poisson
processes. See, for example, the overviews by \citet{bitran2003overview}
or \citet{aviv2012dynamic}.
We stress that in other communities, such as financial markets,
both diffusion processes and jump-diffusions are common for modelling
demand and spot-prices~\citep{benth2014stochastic}. The survey by \citet{carmona2014survey}
gives an indication of how flexible more advanced models used for commodity
markets can be. With this chapter, we wish to inspire the revenue
management community to take advantage of this research when modelling
uncertainty in situations where it may be more relevant than Poisson
processes. For the remainder of this section,
however, we mainly focus on modelling in the revenue management literature.

A pure Poisson process assumption is not compatible with taking a
time-continuum limit for sales volume with demand uncertainty and may
be better suited for demand
modelling in industries with lower product sales volumes, such as
the airline and hotel industries. Our model should therefore scale
better to large-inventory problems, which is more relevant to retail chains.
\citet{maglaras2006dynamic} and \citet{schlosser2015dynamic1,schlosser2015dynamic2}
propose pricing heuristics similar to this
chapter, by considering a deterministic model based on the asymptotic continuum
limit of Poisson processes.
To the author's knowledge, however,
few attempts have been made to unify demand uncertainty with the continuum
limit. For example, \citet{raman1995optimal} and \citet{wu2016dynamic} model
demand uncertainty as increments of a Brownian motion. As we show
in this chapter, their approaches lead to demand processes that admit
negative sales, with a probability approaching $1/2$ over
infinitesimally small time periods. We believe this is an important
factor for why so little research has been done in this area.
This chapter proposes a different
approach to modelling demand uncertainty in order to remedy this.
In our approach, the parameters of the system are described
as diffusion processes that are solutions to
stochastic differential equations (SDEs).
This enables modelling of the demand volatility, which Poisson
processes do not.
Our proposed approach can be combined with parameter estimation
methods already used by retailers
and also extends naturally to multiple products.
Modelling the demand over time as a diffusion process has previously
been done by \citet{chambers1992estimation} at the macroeconomic scale
with UK national data. His focus was on the data assimilation aspect and was not
applied to a setting of optimal control.

The retailer's dynamic pricing policies are stochastic processes that
control the SDE which describes the depletion of stock. In this
chapter, we seek an optimal pricing policy that maximises the expected value of
the profit over a given pricing period.
One way to find an optimal pricing policy is to solve an associated nonlinear partial
differential equation (PDE), known as the Hamilton-Jacobi-Bellman (HJB) equation
\citep{pham2009continuous}.
We provide closed-form solutions for the HJB equation in the
deterministic case, for both linear
and exponential demand functions. The solution identifies two pricing
regimes, one where the retailer maximises their profits without
depleting the inventory and another
where the retailer aims to maximise the price and still deplete
its inventory.
\citet{xu2006monopolistic} have considered a continuous-time pricing
problem where the uncertainty is modelled by a geometric Brownian
motion. Their expected demand function is unbounded, with the result that the
optimal pricing strategy ensures that all stock is sold by the
terminal time. The demand functions we consider are bounded, and by
introducing a penalty on unsold stock we capture the pricing regime
change that does not arise in \citet{xu2006monopolistic}.

In financial markets traders face a similar problem to
that presented in this chapter, known as the optimal execution, or liquidation,
problem. There, a trader tries to sell, or purchase,
a particular amount of an asset by a predetermined time. See, for example,
\citet{cartea2015algorithmic} for an overview of this problem.
Instead of controlling the price, the focus of the retailer pricing
problem, the trader directly controls how much
of the product to sell at a particular time. If the expected demand
model is invertible, the retailer's pricing problem can be
reformulated to control the expected amount of stock to sell at each
time. This reformulation is sometimes chosen in the revenue management
community as well, see, for example, \citet{bitran2003overview}.
In this chapter we focus on the formulation that writes the expected
demand model in terms of the price.

By investigating the terms in the HJB equation,
we identify the cases where the deterministic-case solution is appropriate.
% By numerically investigating the optimal pricing policy for the
% stochastic system, we identify the cases where the deterministic system
% solution is appropriate.
Potentially significant changes to the
pricing policy for the stochastic system are at the interface between
the two pricing regimes --- far away from this interface the
deterministic pricing policy is near-optimal. For example, the
expected price path is decreasing when one takes into account
uncertainty, while it is not for the deterministic heuristic.
For a risk-neutral decision maker, however, the differences in profit
are insignificant for most cases that may be relevant in industry.


The chapter is structured as follows. In \Cref{sec:cts_modelling}, we
describe the modelling of the system and compare the new parameter uncertainty
approach to the existing Brownian increments approach.
Then, a formulation of the pricing problem and the associated HJB
equation is given in \Cref{sec:decision_formulation}. We also propose
a method to estimate the multiplicative factor in our model, in
order to implement the pricing policy in practice.
The optimal pricing policy in the deterministic limit is covered in
\Cref{sec:deterministic_hjb}, and the comparison to the stochastic
system is shown in \Cref{sec:stochastic_hjb}.
Extensions to the problem, such as other models for uncertainty, and
risk aversion, are discussed in \Cref{sec:extensions}.
Finally, we conclude and
suggest avenues for further research in \Cref{sec:cts_conclusion}.

\section{Modelling demand and uncertainty}\label{sec:cts_modelling}
For a given, positive amount of initial stock of a product, we are interested in
modelling the product sales over some finite time period.
Assume that the initial quantity of stock is large, and that there is a
substantial volume sold over time periods that are small compared to the total
period of interest. These assumptions can apply to many products sold
by large retailers, such as lettuce or milk. For example, in the monopoly setting, this leads to
a continuum model similar to that of \citet{kalish1983monopolist}.
For a given product, denote the amount of stock left at time $\hat{t}$ by $\hat{S}(\hat{t})$,
and let $\hat{q}(\hat{a})$ represent product demand at price $\hat{a}$, per unit
time.
For simplicity, we assume $\hat{q}$ does not explicitly depend on time.
In the continuum limit, the change in  stock at time $\hat{t}$
is thus
\begin{equation}
  d\hat{S}(\hat{t})=
  \begin{cases}
    -\hat{q}(\hat{a})\,dt & \textnormal{if } \hat{S}(\hat{t})>0,\\
    0& \textnormal{if }  \hat{S}(\hat{t})\leq 0.
  \end{cases}
\end{equation}
For a pricing policy
$\hat{\alpha}(\hat{t})$, the remaining stock at time $\hat{t}$ is then
\begin{equation}
  \hat{S}(\hat{t})=\hat{S}(0)-\int_0^{\hat{t}}\hat{q}(\hat{\alpha}(\hat{u}))\,d\hat{u}.
\end{equation}
In the remainder of this chapter,
we emphasise the dependence of remaining stock on a particular
pricing policy $\hat{\alpha}(\hat{t})$ using the superscript $\hat{S}^{\hat{\alpha}}$.

At the start of the prediction period, it is not known exactly what
the demand will be at future times. We now discuss how to
represent this uncertainty in the model. First, we note that the Brownian noise
approach of \citet{raman1995optimal} and \citet{wu2016dynamic} leads to negative
sales with probability tending to $0.5$ as the time period goes to zero.
Then, we propose a method that guarantees non-negative sales over all
time periods.
Let $\hat{W}(\hat{t})$ denote a Brownian motion
\citep{oksendal2000stochastic}, and let
$\hat{\sigma}(\hat{t},\hat{s},\hat{a})$ be the volatility
in demand as a function of time, stock, and price. Then
one may say that the uncertainty in future
sales is due to the changes in $\hat{W}(\hat{t})$, in the following sense,
\begin{equation}\label{eq:brownian_noise}
  d\hat{S}^{\hat{\alpha}}(\hat{t}) =
  \begin{cases}
    -\hat{q}(\hat{\alpha}(\hat{t}))\,d\hat{t} +
    \hat{\sigma}(\hat{t},\hat{S}(\hat{t}),\hat{\alpha}(\hat{t}))\,d\hat{W}(\hat{t})
    & \textnormal{if } \hat{S}(\hat{t})>0,\\
    0 &  \textnormal{if } \hat{S}(\hat{t})\leq0.
  \end{cases}
\end{equation}
In \Cref{fig:brownian_paths} we can see six realisations of
$\hat{S}^{\hat{\alpha}}(\hat{t})$ under
this model, using $\hat{q}(\hat{a})=1$ and $\hat{\sigma}(\hat{t},\hat{s},\hat{a})=0.05$. As we zoom in on
the sales paths, it is obvious that the stock often
increases over short time periods, corresponding to negative sales.
\begin{figure}[htbp]
  \centering
  \includegraphics[width=0.5\textwidth,axisratio=1.25]{sde_brownian_sols}%
  \includegraphics[width=0.5\textwidth,axisratio=1.25]{sde_brownian_sols_zoom}
  \caption[Evolutions of stock for a constant demand forecast]{Evolutions of stock $\hat{S}^{\hat{\alpha}}(\hat{t})$ for a constant
    demand forecast $\hat{q}(\hat{a})=1$ with different sample paths $\omega_k$
    of a Brownian motion.
    The right hand figure shows that over small timescales, sales are
    often negative.
    All values are described in dimensionless quantities for
    simplicity, see \Cref{subsec:nondimensionalisation}.
  }\label{fig:brownian_paths}
\end{figure}

We now show that for any model with $\hat{\sigma}>0$, the Brownian noise
model~\eqref{eq:brownian_noise} causes negative sales with high
probability.
For a given time period $[0,\Delta{\hat{t}})$, assume the demand and volatility are constant,
$\hat{q}(\hat{a})=\tilde q>0$ and $\hat{\sigma}(\hat{t},\hat{s},\hat{a})=\tilde \sigma>0$.
Then $\hat{S}(\Delta{\hat{t}})=\hat{S}(0)-\tilde q\Delta{\hat{t}} + \tilde{\sigma} \hat{W}(\Delta{\hat{t}})$.
Sales are negative in this period if $\hat{S}(\Delta{\hat{t}})>\hat{S}(0)$, that is when
$\tilde{q}\Delta{\hat{t}}+\tilde{\sigma} \hat{W}(\Delta{\hat{t}}) < 0$.
The distribution of Brownian motion is normally distributed as
$\hat{W}(\Delta{\hat{t}})\sim \mathcal{N}(0,\Delta{\hat{t}})$. So $\hat{W}(\Delta{\hat{t}})$ is equal in
distribution to $\sqrt{\Delta{\hat{t}}}Z$, where $Z\sim \mathcal{N}(0,1)$, and thus
\begin{align}
  \mathbb{P}(\hat{S}(\Delta{\hat{t}})>\hat{S}(0))
  &=\mathbb{P}(Z\leq -\sqrt{\Delta{\hat{t}}}\,\tilde q / \tilde \sigma)
    \to 0.5^-,& \text{ as } \Delta{\hat{t}} \to 0^+.
\end{align}

We therefore argue that it is generally inappropriate to model the uncertainty
in future demand with increments of a Brownian motion. Instead, the
uncertainty in demand could be modelled more realistically by
describing the evolution of parameters in the demand function.
This chapter focuses on multiplicative demand
uncertainty, which may arise from estimates of seasonality or
long-term trends in demand. As such, we introduce a multiplicative
parameter $\hat{g}\geq 0$, and consider the demand function given by
$({\hat{a}},\hat{g})\mapsto \hat{q}(\hat{a})\hat{g}$.
Assume that the parameter $\hat{g}$ is not directly
affected by price, that occurs through $\hat{q}(\hat{a})$,
but  only represents exogenous information outside the
retailer's control.
For our purposes, we model the multiplicative parameter as
a geometric Brownian motion (GBM),
\begin{equation}
  \hat{G}(\hat{t}) = \exp\left(
    -\frac{1}{2}\hat{\sigma}^2\hat{t} +\hat{\sigma}
    \hat{W}(\hat{t})\right),
\end{equation}
with volatility coefficient
$\hat{\sigma}\geq 0$, with $\hat{\sigma}=0$ corresponding to a
deterministic model. The GBM approach is also considered
by \citet{xu2006monopolistic} for a similar dynamic pricing problem to
that presented in this chapter, but with a different choice of demand function
$\hat{q}(\hat{a})$ and unsold stock cost.
We restrict ourselves to a GBM with no
drift to focus on the impact of the uncertainty, rather than modelling
time-dependent behaviour such as seasonality.
For any $\Delta{\hat{t}}\geq 0$, some relevant properties of $\hat{G}(\hat{t})$ are as follows.
\begin{align}
  \hat{G}(0)
  &=1,\\
  \mathbb{E}[\hat{G}(\hat{t}+\Delta{\hat{t}})\mid \hat{G}(\hat{t})]
  &=\hat{G}(\hat{t}),
  &\text{martingale property,}\\
  \mbox{Var}[\hat{G}(\hat{t}+\Delta{\hat{t}})\mid
  \hat{G}(\hat{t})]&={\hat{G}(\hat{t})}^2(e^{\hat{\sigma}^2
                     \Delta{\hat{t}}}-1), & \text{increasing variance.}
\end{align}
With this model, we expect future demand $\hat{q}(\hat{a})\hat{G}(\hat{t})$ at a given price $\hat{a}\geq
0$ to be the current experienced demand, but with decreasing certainty
the further ahead we forecast.
The SDE governing the system, started at
$\hat{S}(0)>0$, $\hat{G}(0)=1$, is
\begin{align}
  \begin{split}\label{eq:gbm_sde}
    d\hat{G}(\hat{t})&=\hat{\sigma} \hat{G}(\hat{t})\,d\hat{W}(\hat{t}),\\
    d\hat{S}^{\hat{\alpha}}(\hat{t})&=-\hat{q}(\hat{\alpha}(\hat{t}))\hat{G}(\hat{t})\,
    d\hat{t},\qquad\text{stopped at zero}.
  \end{split}
\end{align}
The sample paths of $\hat{S}^{\hat{\alpha}}(\hat{t})$ following the GBM model, as shown in
\Cref{fig:sde_gbm_const_x}, are more regular
than of the Brownian noise model.
\begin{figure}[ht]
  \centering
  \includegraphics[width=0.5\textwidth,axisratio=1.25]{sde_gbm_const_x}
  \caption[Sample paths of stock following the GBM model]{Sample paths of $\hat{S}^{\hat{\alpha}}(\hat{t})$ following the GBM
    model~\eqref{eq:gbm_sde}, with $\hat{q}(\hat{\alpha}(\hat{t}))=1$, and GBM volatility
    $\hat{\sigma} = 0.1$. The paths are  more regular than that of the
    Brownian noise model shown in \Cref{fig:brownian_paths}.
  }\label{fig:sde_gbm_const_x}
\end{figure}

\subsection{Non-dimensionalised system}\label{subsec:nondimensionalisation}
In order to capture similarities between different pricing decisions, irrespective of units
such as a particular currency, it is helpful to work in a dimensionless system.
We thus non-dimensionalise the model, which also helps to reduce the
number of parameters in the decision problem.
The units at play in our system are the time and the product price,
for example measured in weeks and \pounds.
If $\hat{s},\hat{a}$, $\hat{t}$ denotes unscaled quantities of stock, price, and
time respectively, we rescale them with dimensionless quantities
\begin{align}
  s&=\frac{\hat{s}}{\hat{S}(0)},&t &=\frac{\hat{t}}{\hat{T}},
  &a &= \frac{\hat{a}}{\bar{a}}.
\end{align}
Here, $\hat{S}(0)\gg 1$ is the initial quantity of stock, $\hat{T}$ is the time-horizon
for the pricing problem, and $\bar{a}$ is some reference price chosen
to make typical prices $a$ continuous and of order one.
This scaling means that we work solely with stock and time on the unit
interval, $s,t\in[0,1]$.
For typographical reasons we define the upper limit of the time
horizon as $T$, which means $T=1$ and that
\begin{equation}
  t\in[0,T].
\end{equation}
In order to write down a dimensionless formulation of the system's
SDE~\eqref{eq:gbm_sde}, we need to work with the expected demand
function, volatility parameter, and Brownian motion defined as
\begin{align}
  q(a)&=\frac{\hat{T}}{\hat{S}(0)}\hat{q}(a \bar{a}),
  &\sigma &= \sqrt{\hat{T}}\hat{\sigma},
  &W(t) &= \frac{\hat{W}(\hat{t})}{ \sqrt{\hat{T}}}.
\end{align}
It follows that the multiplicative term is given by
\begin{align}
  \hat{G}(\hat{t})
  &=\exp\left(
    -{\textstyle\frac{1}{2}}\hat{\sigma}^2\hat{T}t +\hat{\sigma} \sqrt{\hat{T}}\,W(t)\right)\\
  &=\exp\left(-{\textstyle\frac{1}{2}}\sigma^2t +\sigma W(t)\right).
\end{align}
So we define dimensionless versions of the process $\hat{G}(\hat{t})$
using
\begin{equation}
  G(t)=\hat{G}(\hat{T}t),\quad g=\hat{g}.
\end{equation}
Thus, for a given pricing policy $\alpha(t)$, the
system starts at $S(0),G(0)=1$, and evolves according to
\begin{align}
  \begin{split}\label{eq:gbm_sde_nondim}
    d G(t)&=\sigma G(t)\,dW(t),\\
    dS^{\alpha}(t)&=-q(\alpha(t))G(t)\,dt,
    \qquad\text{stopped at zero}.
  \end{split}
\end{align}

\section{Formulation of the decision
  problem}\label{sec:decision_formulation}
We restate the retailer's objective which
guides the choice of pricing policy: Over some decision horizon, continuously
adjust the price of a given product in order to maximise
the total profit generated from sales revenue minus the cost of handling
unsold items at the terminal time. We formulate this mathematically as
a continuous-time stochastic optimal control problem, for which the optimal pricing
policy can be found by solving the associated HJB equation.
In \Cref{subsec:param_estim} we address the real-world restrictions of
the continuous-time assumption.

Let $C>0$ denote the handling cost per unit of stock at the terminal
time. Define $T_h$ to be the hitting time $T_h=\min \{T, T_0\}$, where
$T_0=\inf\{t\geq 0\mid S^\alpha(t)=0\}$.
The total profit accrued from a pricing policy $\alpha(t)$ is then
\begin{equation}\label{eq:profit_expression}
  P(\alpha) = \int_0^{T_h}\alpha(u)q(\alpha(u))G(u)\,du - CS^\alpha(T).
\end{equation}
Note that we focus on time horizons where we believe discounting
future cash is negligible.
This profit is a random variable that depends on the event $\omega$
or, equivalently, the path of the
Brownian motion $W(t)$. %and the pricing policy $\alpha(t)$.
In the context of this chapter, we assume a retailer that
wants to maximise the expected profit $\mathbb{E}[P(\alpha)]$.

Further, we restrict the product price to be in some closed  interval
$A\subset\mathbb{R}_{\geq 0}$.
In addition, we only look for Markovian pricing policies in an
admissible set $\mathcal{A}$. We say that $\alpha(t)$ is Markovian if
there is a function of the form $a(t,s,g)$ such that for each event $\omega$,
\begin{equation}
  \alpha(t)(\omega) = a(t,S^\alpha(t)(\omega), G(t)(\omega)) \in A.
\end{equation}
Thus, we seek pricing policies that set the price at time $t$, based
on the knowledge of the state at that time. We also assume that
$\mathcal{A}$ only contains pricing policies such that the integral
in~\eqref{eq:profit_expression} exists and
$P(\alpha)$ is integrable. We can now state the mathematical problem we
seek to solve in the remainder of the chapter.
\begin{mydef}[Pricing problem]
  In order to maximise the retailer's expected profit, find a solution to
  the stochastic optimal control problem
  \begin{equation}\label{eq:pricing_problem}
    \max_{\alpha\in \mathcal{A}}\,\mathbb{E}[P(\alpha)].
  \end{equation}
\end{mydef}
Our chosen strategy for finding the corresponding pricing function $a^*(t,s,g)$
of a maximiser $\alpha^*\in\mathcal{A}$ is to solve the associated HJB
equation for the pricing problem.
For a detailed explanation of the theory behind stochastic optimal control and
HJB equations, see, for example, \citet{pham2009continuous}.
The HJB equation is a nonlinear PDE, where the solution describes the
value of being in a particular state. We assume that there exists a
solution to the HJB equation but do not worry about
uniqueness in this chapter. From the
value function defined below, one can
calculate the optimal pricing function $a^*(t,s,g)$.

If at time $t$, we know the value of $S(t)$ and $G(t)$, then the value
function represents the expected value of applying the optimal pricing
policy for the remainder of the pricing period.
We define the value function $v(t,s,g)$ for $t \leq T$ and $s,g\geq
0$ by
\begin{align}
  v(t,s,g) &=
             \max_{\alpha \in \mathcal{A}}\,
             \mathbb{E}_{t}\left[
             \int_t^{T_h}\alpha(u)q(\alpha(u))G(u)\,du-CS^\alpha(T)
             \right],\label{eq:def_value_interior}\\
  v(t,0,g)&= 0,\\
  v(t,s,0)&=-Cs.
\end{align}
The subscript on the expectation denotes that we condition on
$S^\alpha(t)=s$ and $G(t)=g$.
In the limit $t\to T$, we see from~\eqref{eq:def_value_interior} that
$v(T,s,g)=-Cs$.

We derive the HJB equation for $v$ in a non-rigorous manner
using a
continuous-time analogue of the Dynamic Programming principle
stated in \Cref{subsec:adp_bellman_equation}. For a rigorous proof,
see, for example, \citet[Ch.~3]{pham2009continuous}.
Let $\Delta{t}>0$ and, for brevity, denote
\begin{equation}
  v^{(\Delta{t})} =
  v(t+\Delta{t},S^\alpha(t+\Delta{t}),G(t+\Delta{t}))
\end{equation}
Then, as in~\eqref{eq:dynamic_programming_discrete} from
\Cref{subsec:adp_bellman_equation}, we can
split~\eqref{eq:def_value_interior} into two terms
\begin{align}
  v(t,s,g) = \max_{\alpha\in\mathcal{A}}\,
  \mathbb{E}_{t}\left[
  \int_t^{t+\Delta{t}} \alpha(u)q(\alpha(u))G(u)\,du
  + v^{(\Delta{t})}\right]
\end{align}
using the tower rule $\mathbb{E}_t[\cdot] = \mathbb{E}_t[\mathbb{E}_{t+\Delta{t}}[\cdot]]$.
It follows that
\begin{equation}
  % \label{eq:cts_bellman_principle}
  % \max_{\alpha\in\mathcal{A}}\,
  % \mathbb{E}_{t}\left[
  %   \int_t^{t+\Delta{t}} \alpha(u)q(\alpha(u))G(u)\,du
  %   + v^{(\Delta{t})}-v(t,s,g)\right]
  % &=0, &&\textnormal{or}\\
  \label{eq:cts_bellman_principle_ineq}
  \mathbb{E}_{t}\left[
    \int_t^{t+\Delta{t}} \alpha(u)q(\alpha(u))G(u)\,du
    + v^{(\Delta{t})}-v(t,s,g)\right]
  \leq 0\quad  \forall \alpha\in\mathcal{A},
\end{equation}
with equality achieved for a maximiser $\alpha^*\in\mathcal{A}$.
Consider an arbitrary $\alpha\in\mathcal{A}$ and let $a=\alpha(t)\in
A$ denote its value at time $t$.
We have the following Taylor-expansion of the
integrand $\alpha(u)q(\alpha(u))G(u)$.
\begin{equation}
  \mathbb{E}_{t}\left[
    \int_t^{t+\Delta{t}}
    \alpha(u)q(\alpha(u))G(u)\,du\right]
  = \Delta{t}\,aq(a)g  + \mathcal{O}({\Delta{t}}^2)
\end{equation}
By It\^{o}'s lemma \citep[Ch.~1]{pham2009continuous}, a
generalisation of the fundamental theorem of calculus, we know that
\begin{align}
  \mathbb{E}_t[v^{(\Delta{t})}]-v(t,s,g)
  &= \mathbb{E}_t  \int_t^{t+\Delta{t}}
    (v_t
    +{\textstyle\frac{\sigma^2g^2}{2}}
    v_{gg}
    -q(\alpha(u))g v_s)(u,S^{\alpha}(u),G(u))
    \,du\\
  &= \Delta{t}\left( v_t
    +{\textstyle\frac{\sigma^2g^2}{2}}
    v_{gg}
    -q(a)g v_s \right)(t,s,g) + \mathcal{O}({\Delta{t}}^2).
\end{align}
The subscripts on $v$ denote partial derivatives with respect to the
given argument.
Introducing the Taylor expansion into~\eqref{eq:cts_bellman_principle_ineq}
yields
\begin{equation}
  v_t
  +{\textstyle\frac{\sigma^2g^2}{2}}
  v_{gg}+
  g(a-v_s)q(a)
  +\mathcal{O}(\Delta{t})\leq 0,\qquad \forall a\in A.
\end{equation}
The upper bound is reached for a maximiser $a^*=\alpha^*(t)$, and thus
taking $\Delta{t}\to 0$ gives us the the local representation
\begin{equation}
  v_t+\frac{\sigma^2g^2}{2}v_{gg}
  + g(a^*-v_s)q(a^*) = 0,
\end{equation}
valid for $t<T$ and $s,g>0$.
Together with the boundary and terminal conditions on
$v$, this constitutes the HJB equation for the pricing problem,
\begin{align}\label{eq:hjb_interior}
  v_t(t,s,g)+\frac{\sigma^2}{2} g^2v_{gg}(t,s,g)
  +g\max_{a\in A}\{(a-v_s(t,s,g))q(a)\} &= 0,\\ %, \quad
  % (t,s,g)\in{(0,1)}^2\times\mathbb{R}_{>0}.
  v(t,0,g) &= 0,\\
  v(t,s,0) &= -Cs,\\
  v(T,s,g) &= -Cs.
\end{align}

If we know $v$, the optimal pricing policy
$\alpha^*(t)=a(t,S^{\alpha}(t),G(t))$
can be calculated from the univariate optimisation problem
\begin{equation}\label{eq:pricing_function_general}
  a^*(t,s,g) = \argmax_{a\in A}\{(a-v_s(t,s,g))q(a)\}.
\end{equation}
In \Cref{sec:deterministic_hjb} and \Cref{sec:stochastic_hjb}, solutions for the
pricing problem are found via the HJB equations with linear and
exponential demand functions.
\begin{remark}\label{rem:viscosity}
  The value function is not necessarily sufficiently smooth to satisfy
  the HJB equation~\eqref{eq:hjb_interior} in the classical sense.
  This is indeed the case for some examples in this chapter when $\sigma=0$,
  and we must therefore consider
  the solutions in the viscosity sense. See \citet{pham2009continuous}
  for a description of viscosity solutions to HJB equations.
\end{remark}

\subsection{Parameter estimation}\label{subsec:param_estim}
In a real retail application we cannot update the price in continuous
time, and it is not possible to infer $G(t)$ exactly.
To represent real world conditions, we assume that
the price is piecewise constant and updated frequently at fixed time points
$t_0<t_1<\cdots<T_h$.
Further, we assume that stock levels are only observed at these time
points. When computing an optimal pricing strategy
we still consider the continuous-time
function~\eqref{eq:pricing_function_general} that can be computed from
the HJB equation as
the optimal pricing function for the problem. Such a continuous-time
approximation to inherently discrete systems is common, for example
in pricing of options using the Black-Scholes equations
\citep{black1973pricing}.
In order to use a pricing function $a(t_k,s,g)$ we must estimate
$G(t_k)$.
Due to the Markov properties of $G(t)$ and $S^{\alpha}(t)$, information
about the process for $t<t_{k-1}$ is not needed, and we can estimate
$G(t_k)$ based on $\alpha(t_{k-1})$, $S^\alpha(t_{k-1})$, and $S^\alpha(t_k)$.
For brevity, let us leave out the superscript $\alpha$ of
$S^\alpha(t)$ for the remainder of this section.
By assumption, the price has been constant, $\alpha(u)=a_{k-1}$, for the time
period $u\in[t_{k-1},t_{k})$. Say $S^\alpha(t_{k})>0$, and that we wish to update the price at time
$t_{k}$.
From~\eqref{eq:gbm_sde_nondim}, the SDE describing our system, we have
\begin{equation}\label{eq:s_integrated}
  S(t_{k})=S(t_{k-1})-q(a_{k-1})\int_{t_{k-1}}^{t_{k}}G(u)\,du.
\end{equation}
In the numerical examples in this chapter, we estimate $G(t_{k})$ with
$\hat G(t_{k})=\frac{S(t_{k-1})-S(t_{k})}{q(a_{k-1})(t_{k}-t_{k-1})}$.
We now discuss the derivation and properties of this estimator.

Define $B(t) =  \exp(-\sigma^2 t/2+\sigma\sqrt{t}Z_{k-1})$, where $Z_{k-1}\sim
\mathcal{N}(0,1)$. The evolution of $G(t)$ is known in closed form,
which gives
$G(t_{k})=G(t_{k-1})B(\Delta t)$, with $\Delta t = t_{k}-t_{k-1}$.
From~\eqref{eq:s_integrated} it follows that
\begin{align}
  G(t_{k}) = \frac{S(t_{k-1})-S(t_{k})}{q(a_{k-1})}\frac{B(\Delta
  t)}{\int_{0}^{\Delta t}B(u)\,du}.
\end{align}

So long as $\Delta t=t_{k}-t_{k-1}$, and the variance of $B(u)$, are
sufficiently small, we may use the approximation
$\int_{0}^{\Delta{t}}B(u)\,du\approx  \frac{\Delta
  t}{2}(B(0)+B(\Delta t))=\frac{\Delta
  t}{2}(1+B(\Delta t))$.
Hence, the conditional distribution of $G(t_{k})$ is approximated as
\begin{align}
  G(t_{k})\mid (a_{k-1},S(t_{k-1}),S(t_{k}))
  &\approx
    \frac{S(t_{k-1})-S(t_{k})}{q(a_{k-1})}\frac{2}{\Delta t}\frac{B(\Delta t)}{1+B(\Delta t)}\\
  &\approx \frac{S(t_{k-1})-S(t_{k})}{q(a_{k-1})}\frac{1+B(\Delta t)}{2\Delta t}.
\end{align}
The second approximate equality comes from the Taylor
expansion $\frac{x}{1+x}\approx \frac{1}{4}(1+x)$ about $x=1$.
This gives us the following expressions for the first two moments
of $G(t_{k})$:
\begin{align}\label{eq:conditional_mean_g}
  \mathbb{E}[G(t_{k})\mid a_{k-1},S(t_{k-1}),S(t_{k})]
  &\approx \frac{S(t_{k-1})-S(t_{k})}{ q(a_{k-1})\Delta t},\\
  \mbox{Var}[G(t_{k})\mid a_{k-1},S(t_{k-1}),S(t_{k})]
  &\approx \frac{1}{4}{\left(\frac{S(t_{k-1})-S(t_{k})}{
    q(a_{k-1})\Delta t}\right)}^2\left( e^{\sigma^2\Delta t} -1 \right).
\end{align}
The conditional expectation in~\eqref{eq:conditional_mean_g} is equal
to our estimator $\hat G(t)$.
\Cref{fig:gbm_estimate} shows the distribution of the relative
difference $1-\hat G(t)/G(t)$ between
the estimate and the true $G(t)$, for
the parameters used in \Cref{sec:stochastic_hjb}. The relative
estimator error when $\sigma=0.1$ and $\Delta t=0.01$
is typically within \SI{1}{\percent}.
\begin{figure}[htbp]
  \centering
  \includegraphics[width=0.6\textwidth,axisratio=1.55]{gbm_estimate}
  \caption[Distribution of the relative error in estimating $G(t)$]{Distribution of the relative error in estimating $G(t)$ with
    $\sigma=0.1$ and $\Delta t= 0.01$, based on \num{100000} samples.}\label{fig:gbm_estimate}
\end{figure}

\section{Solution to the deterministic
  system}\label{sec:deterministic_hjb}
In this section, we provide solutions to the pricing problem in the
deterministic case, $\sigma = 0$, for families of linear and exponential
demand functions $q_l(a)$ and $q_e(a)$ respectively.
\begin{align}
  q_l(a)&=q_1-q_2a,&\text{for } q_1,q_2>0,\\
  q_e(a)&=q_1e^{-q_2a},&\text{for } q_1,q_2>0.
\end{align}
These demand functions are often used in the literature. For a discussion
about their properties and usage in modelling demand, see \citet[Ch.~7]{talluri2006theory}.
With both demand functions, the optimal pricing function policy is to hold the
price constant over the pricing horizon. They translate to the
following intuition: For small amounts of stock, relative to demand,
sell at the highest price so that all stock is depleted by the
terminal time. As the amount of stock increases, the retailer should
decrease the price until it arrives at the price that maximises the
profits balancing revenue and cost of unsold stock at the terminal
time.

\subsection{Linear demand function}
When $q(a)=q_1-q_2a$ for $q_1,q_2>0$, the maximisation in the HJB
equation can be solved in closed form.
To reduce the number of parameters, we can rescale the price per
product with $q_2$ by setting $q_2a$ to $a$.
Then $q(a)=q_1-a$, and we set the pricing interval to $A=[0,q_1]$ so that the demand
function is non-negative.
Now, use the ansatz that $a^D(t,s,g)$ sets the price so that
$\alpha^D(t)$ is
constant for the remainder of the pricing period.
From the expression of the value function
in~\eqref{eq:def_value_interior}, the ansatz pricing policy must
also be given by
\begin{align}
  a^D(t,s,g)
  &=\argmax_{a\in A}\left\{\int_t^Taq(a)g\,du -
    C\left(s-\int_t^Tq(a)g\,du\right) \,\big|\,  s \geq \int_t^Tq(a)g\,du\right\}\\
  &=\argmax_{a\in A}\left\{(T-t)(a+C)q(a)g-Cs \mid  s \geq
    (T-t)q(a)g\right\}.
\end{align}
The maximiser therefore satisfies the equality constraints that
$s$ equals $(T-t)gq(a)$, is zero, or is in the interior of the feasible set,
given by $A$.
For $q(a)=q_1-a$, we can verify that this implies
\begin{equation}\label{eq:astar_linear}
  a^D(t,s,g)=\begin{cases}
    q_1-\frac{s}{(T-t)g}, &\text{if } 0\leq s\leq
    (T-t)g\min\{q_1,\frac{1}{2}(q_1 +C)\},\\
    \max(0,q_1-C)/2,&\text{otherwise.}
  \end{cases}
\end{equation}
It follows that $T_h=T$ when following the optimal pricing strategy.
This is an intuitive result, because if $T_h<T$, one can increase the
price until $T_h=T$ and $S^\alpha(T)=0$, which earns extra revenue at no extra cost.
The pricing policy suggested by the
deterministic assumption provides the following, obvious, heuristics:
First, find the price that maximises profits, ignoring inventory
constraints. Second, if the sales forecast suggests that you will deplete stock
before the end of the time horizon at this price, increase it
accordingly. This is consistent with the heuristic proposed
by \citet{schlosser2015dynamic1,schlosser2015dynamic2}, which
he finds by considering a deterministic continuum
approximation to a Poisson process demand model.

\begin{example}\label{ex:acecplot}
  To demonstrate what form the pricing function may take,
  \Cref{fig:hjbsol_a_cec} shows a plot of $a^D(t,s,1)$ for a given set of
  parameters. We have chosen the parameters so that the pricing
  problem starts at the most interesting point: at the kink separating
  the regimes where all the stock is sold out and where it is not.
  This corresponds to a combination of parameters such that
  $\min(q_1,\frac{1}{2}(q_1+C))=1$.

  \begin{figure}[htbp]
    \centering
    \includegraphics[width=0.6\textwidth,axisratio=1.15]{hjbsol_a_cec}
    \caption*{$a^D(t,s,1)$}
    \caption[Example optimal, deterministic pricing function]{Example optimal, deterministic pricing function $a^D(t,s,1)$
      from~\eqref{eq:astar_linear}, as a function of time and stock.
      Notice how sensitive $a^D(t,s,1)$ is to changes in $s$, for large
      $t$.
      The parameters used are $q_1=3/2$, and $C=1/2$.
      In \Cref{fig:hjbsol_a} we compare this function to the optimal price
      when $\sigma=0.1$.
    }\label{fig:hjbsol_a_cec}
  \end{figure}
\end{example}

To verify that $a^D(t,s,g)$ given by~\eqref{eq:astar_linear} is
the solution to the deterministic
pricing problem, we show that it satisfies the HJB equation.
On the interior of the $(t,s,g)$-domain, $a^D(t,s,g)$ must solve
\begin{equation}
  \max_{a\in A} \{(a-v_s(t,s,g))(q_1-a)\}.
\end{equation}
Let $\mathcal{P}_A$ denote the projection of the real line to $A$. The
objective is concave, and hence the maximiser is
\begin{equation}\label{eq:astar_hjb_linear}
  a^D(t,s,g) = \mathcal{P}_A\left[\frac{q_1+v_s}{2}\right].
\end{equation}
Define $\Gamma$ to be the boundary of the terminal time problem,
that is when $t=T$, $s=0$, or $g=0$.
Let us assume that $C< q_1$, then the
deterministic-case HJB equation for $v$ is
\begin{align}\label{eq:hjb_linear}
  v_t(t,s,g)+
  \frac{g}{4}{( q_1-v_s(t,s,g))}^2
  &=0, \\%&(t,s,g)&\in{(0,1)}^2\times \mathbb{R}_{>0},\\
  v(t,s,g) &= -Cs,&(t,s,g)&\in \Gamma.
\end{align}
If $C\geq q_1$, then there are regions where $a^D(t,s,g)=0$. In
particular, this means that for $t,s,g$ such that
$s>(T-t)gq_1$, $v$ must satisfy
\begin{equation}\label{eq:hjb_linear_Clarge}
  v_t(t,s,g) - gq_1v_s(t,s,g)=0.
\end{equation}

From the two expressions for $a^D(t,s,g)$ in~\eqref{eq:astar_linear}
and~\eqref{eq:astar_hjb_linear}, the ansatz implies that the value
function must satisfy
\begin{align}\label{eq:v_func_linear_det}
  v^D(t,s,g)
  &=\begin{cases}
    q_1s-\frac{s^2}{(T-t)g},
    &\text{if } 0\leq s\leq
    (T-t)g\min\{q_1,\frac{1}{2}(q_1 +C)\},\\
    -Cs + V(C)(T-t)g,
    &\text{otherwise.}
  \end{cases}\\
  V(C)
  &= \begin{cases}
    {\left[\frac{1}{2}(q_1+C)  \right]}^2,&C<
    q_1,\\
    q_1C,&C\geq q_1.
  \end{cases}
\end{align}
This $v$ does indeed satisfy the deterministic HJB equation given
by~\eqref{eq:hjb_linear}-\eqref{eq:hjb_linear_Clarge}, and hence we
can conclude that $a^D(t,s,g)$ is an optimal pricing function. Note that $v$ is
not smooth for all parameter combinations, and is therefore considered
a solution in the viscosity sense, as noted in \Cref{rem:viscosity}.

\subsection{Exponential demand function}
With the same ansatz that was used for the linear demand function, we
can also find the optimal, deterministic, pricing function for
exponential demand $q(a)=q_1e^{-q_2a}$.
Indeed, this is true for any demand function for which a closed form
solution exists for $\max_{a\in A}\{(a+C)q(a))\}$ and $(T-t)gq(a)=s$.
As with the linear demand, we can eliminate the parameter $q_2$ so that
$q(a)=q_1e^{-a}$, given by replacing $q_2a$ by $a$.
In the exponential demand case, with $A=[0,\infty)$, the ansatz gives
us the optimal pricing function
\begin{equation}\label{eq:afun_exp_cec}
  a^D(t,s,g)=\begin{cases}
    \log{\frac{q_1g(T-t)}{s}},
    &\text{if } 0\leq\frac{s}{q_1g(T-t)}\leq e^{C-1},\\
    \max\left(0,1-C\right),&\text{otherwise}.
  \end{cases}
\end{equation}
For completeness, we state the HJB equation for the exponential demand
case when $C<1$, and provide the solution so that the optimality of~\eqref{eq:afun_exp_cec} can
be verified. The maximiser of $\max_{a\in A}\{(a-v_s)q(a)\}$ in the
HJB equation is $a^D =  1+v_s$. Thus, the value function
must satisfy the HJB equation
\begin{align}
  v_t(t,s,g)
  +gq_1e^{-1-v_s(t,s,g)}
  &=0, \\%&(t,s,g)&\in{(0,1)}^2\times \mathbb{R}_{>0},\\
  v(t,s,g) &= -Cs,&(t,s,g)&\in \Gamma.
\end{align}
The viscosity solution of the HJB equation, acquired from the ansatz $a^D(t,s,g)$
in~\eqref{eq:afun_exp_cec}, is
\begin{align}
  v^D(t,s,g)
  &=\begin{cases}
    s \log \frac{q_1g(T-t)}{s},
    &\text{if } 0\leq \frac{s}{q_1g(T-t)}\leq e^{C-1},\\
    -Cs+V(C)(T-t)g,&\text{otherwise},
  \end{cases}\\
  V(C)&=q_1e^{C-1}.
\end{align}

\section{Impact of uncertainty}\label{sec:stochastic_hjb}
We now discuss to what degree multiplicative uncertainty changes
our policy. In this section, we solve the HJB equation numerically with the linear
demand function $q(a)=q_1-a$ defined on $A=[0,q_1]$, and compare the
resulting pricing policy to the
deterministic-system policy from the previous section.
With the diffusion term in the HJB equation, one can
expect the kink in the deterministic pricing function to smooth
out. It turns out that the difference between an optimal
policy and a heuristic policy based on the solution
to the deterministic system is at most
$\mathcal{O}(\sigma \sqrt{T-t})$. Further, numerical tests indicate that the
closed-form pricing functions found in the previous section
perform sufficiently well in most situations.

We assume in the following that $(q_1+v_s)\in[0,2]$ for
$t\in[0,T)$, $s,g>0$, so
that the pricing function satisfies $a(t,s,g) = (q_1+v_s(t,s,g))/2$ as
given by~\eqref{eq:astar_hjb_linear}.
Using~\eqref{eq:hjb_interior} and~\eqref{eq:hjb_linear} it then
follows that $v$ should satisfy
\begin{align}\label{eq:hjb_impact_uncertainty}
  v_t(t,s,g)+\frac{\sigma^2}{2} g^2v_{gg}(t,s,g)
  +\frac{g}{4}{(q_1-v_s)}^2
  &= 0, \\%&(t,s,g)&\in{(0,1)}^2\times \mathbb{R}_{>0},\\
  v(t,s,g) &= -Cs,&(t,s,g)&\in \Gamma.
\end{align}

The numerical solution to the HJB equation is solved with the
following procedure:
\begin{enumerate}
\item Reformulate the PDE with the similarity
  transformation $\xi = s/g$ and $v = g\phi$.
\item Truncate the boundary
  for $\xi\to\infty$ and set an asymptotic Dirichlet boundary condition
  based on the deterministic-system solution.
\item Approximate
  the PDE for $\phi(t,\xi)$ with central finite differences and the
  \texttt{Tsit5} time stepping procedure in
  \texttt{DifferentialEquations.jl}
  \citep{rackauckas2017differentialequations}, implemented in the Julia
  programming language \citep{bezanson2017julia}.
\end{enumerate}
We denote the computed pricing function and pricing
policy by $a^B(t,s,g)$ and $\alpha^B(t)$ respectively.


\begin{example}\label{ex:function_impact_uncertainty}
  Let us consider the particular example system used in
  \Cref{ex:acecplot}. %\Cref{ex:acecplot}.
  That is, a linear demand function $q(a)=q_1-a$, with $q_1=3/2$
  and $C=1/2$.
  We set the volatility level of $G(t)$ to $\sigma=0.1$, which
  corresponds to a true demand near the terminal time within \SI{20}{\percent} of the
  expected demand $q(a)$, with probability $0.95$.
  \Cref{fig:hjbsol_a} shows the
  optimal pricing function for $g=1$, and a plot of the difference
  $a^B(t,s,g)-a^D(t,s,g)$. The only visible difference is along the kink
  line, $s=g(T-t)$, where $a^B$ smooths out the transition between
  the two regions, and hence sells the product at a slightly higher
  price.
  \begin{figure}[p]
    \centering
    \begin{subfigure}[b]{0.5\textwidth}
      \includegraphics{hjbsol_a_hjb}
      \caption{$a^B(t,s,1)$}
    \end{subfigure}%
    \begin{subfigure}[b]{0.5\textwidth}
      \includegraphics{hjbsol_a_cec_hjb}
      \caption{$(a^D-a^B)(t,s,1)$}
    \end{subfigure}%
    \caption[Comparison of the optimal pricing function to the
    deterministic-case function]{The optimal pricing function ($a^B(t,s,g)$, left) smooths out the
      kink as compared to the deterministic heuristic ($a^D(t,s,g)$,
      \Cref{fig:hjbsol_a_cec}).
      The right plot shows the impact of uncertainty on the optimal
      pricing function: (i)~When we do not expect to sell out of the
      product, the prices are the same. (ii)~When we expect to sell out
      of the product, the deterministic heuristic takes a slightly larger
      price.
      (iii)~In the transition between the two regions, uncertainty
      increases the optimal price.
    }\label{fig:hjbsol_a}
  \end{figure}
\end{example}


\subsection{Asymptotic analysis}\label{subsec:asymptotic_analysis}
\Cref{fig:hjbsol_a} indicates that one can do an asymptotic analysis
of the impact of $0<\sigma \ll 1$ on the pricing
function. \Cref{fig:hjb_cec_zoom} provides another visualisation
of the differences between $a^B$ and $a^D$ to aide the analysis.
\begin{figure}[p]
  \centering
  \begin{subfigure}[b]{0.5\textwidth}
    \includegraphics[width=\textwidth,axisratio=1.2]{hjb_cec_zoom_02_05}
    \caption{$t = 0.5$}
  \end{subfigure}%
  \begin{subfigure}[b]{0.5\textwidth}
    \includegraphics[width=\textwidth,axisratio=1.2]{hjb_cec_zoom_02_07}
    \caption{$t=0.3$}
  \end{subfigure}
  \caption[A zoomed-in comparison of the Bellman and
  deterministic-case functions]{A zoomed in comparison of $a^B(t,s,g)$ and $a^D(t,s,g)$
    in the transformed coordinate $\xi=s/g$.
    The difference between $a^B$ and $a^D$ is of order
    $\sigma\sqrt{T-t}$ at the kink, and
    $-\sigma^2\xi/2$ as we move to the left in the
    plots.
    This example uses $q_1=3/2$, $C=1/2$, and $\sigma=0.2$.
  }\label{fig:hjb_cec_zoom}
\end{figure}

We summarise the results before showing the details of the asymptotic expansions.
Define $\beta=\min(q_1,(q_1+C)/2)$.
There is an inner layer around the surface $\frac{s}{g}=(T-t)\beta$ which
smooths out the kink of the solution $a^D(t,s,g)$ that arises when
$\sigma=0$. At the kink, the first-order correction in the pricing
function is of order $\mathcal{O}(\sigma\sqrt{T-t})$.
The width of the layer is of the order
$\mathcal{O}(\sigma{(T-t)}^{3/2})$, which connects the two pricing
regimes identified in the deterministic case.
As we move from the inner layer to larger values of
$\xi:=s/g>(T-t)\beta$, the solution
tends to the value $\delta=\max(0,q_1-C)/2$, which coincides with
$a^D$. As we move from the inner layer to smaller values of $\xi$,
the leading order solution of the inner layer coincides with $a^D$.
Further, there is a second order correction in the region
$\xi<(T-t)\beta$ equal to $-\sigma^2 \xi/2$.

The first and second order corrections reflect the insights one can
arrive at when taking into account the deviations of actual future
demand from expected demand.  At the interface where we expect to sell
all the inventory at the optimal lower-bound, ``infinite-inventory''
price $\delta$, taking into account the possibility of higher future
demand means that we can increase the price.  When the inventory is so
low that we expect to sell it all at some price higher than $\delta$,
taking into account the possibility of lower future demand means that
we should price the product lower than in the deterministic-case in
order to reduce the probability of having excess inventory at the
terminal time.

We carry out the asymptotic analysis of the pricing function
that arises from the linear demand HJB
equation~\eqref{eq:hjb_impact_uncertainty} under the assumptions of
\Cref{sec:stochastic_hjb}.
Further, we assume that the value and pricing functions are sufficiently
differentiable to carry out the operations below. The smoothing effect
of the diffusion in the PDEs when $\sigma>0$ justifies this assumption.

Let us first reduce the dimension of the problem with a similarity
transform working in reverse time, and then present a PDE satisfied by the pricing function.
Consider the transformations
\begin{align}
  \xi(s,g) &= s/g,\\
  \tau(t)&=T-t,\\
  v(t,s,g)&=g\,\phi(\tau(t),\xi(s,g)),
\end{align}
which require $\phi(\tau,\xi)$ to satisfy
\begin{align}\label{eq:hjb_value_transform}
  \phi_\tau &= \frac{1}{2}\sigma^2\xi^2\phi_{\xi\xi} +
              \frac{1}{4}{(q_1 - \phi_\xi)}^2 \\
  \phi(0,\xi) &= -C\xi\\
  \phi(\tau,0)& = 0,\\
  \phi_\xi(\tau,\infty) &= -C.
\end{align}
In the new coordinates, define the pricing function in terms of the
function $\psi(\tau,\xi)$ so that $a^B(t,s,g)=\psi(\tau(t),\xi(s,g))$.
Then $\psi = (q_1+\phi_\xi)/2$ satisfies the following PDE
that arises from differentiating~\eqref{eq:hjb_value_transform} with
respect to $\xi$.
\begin{align}\label{eq:hjb_policy_transform}
  \psi_\tau&= \frac{1}{2}\sigma^2 \xi^2
             \psi_{\xi\xi} +
             (\sigma^2\xi - q_1 + \psi)\psi_\xi,\\
  \psi(\tau,0) &= q_1,\\
  \psi(\tau,\infty) &= \delta:=\max(0,(q_1-C)/2).
\end{align}
When $\sigma=0$, we know from \Cref{sec:deterministic_hjb} that
the viscosity solution $\psi^{(0)}$ is
\begin{equation}\label{eq:psi_leading}
  \psi^{(0)}(\tau,\xi)=\begin{cases}
    q_1-\xi/\tau, &\text{if } 0\leq \xi\leq
    \beta \tau,\\
    \delta,&\text{if } \hfill \xi > \beta \tau,
  \end{cases}
  \qquad \textnormal{where } \beta:=\min(q_1,(q_1+C)/2).
\end{equation}
This leads us to consider an inner layer asymptotic analysis near
the kink $\xi = \beta \tau$. Zoom in near the kink using the
coordinates
\begin{align}
  x(\tau,\xi) &= (\xi-\beta \tau)/\sigma,\\
  \psi(\tau,\xi) &= \sigma
                   u(\tau,x(\tau,\xi)) + \delta.
\end{align}
In the new coordinates~\eqref{eq:hjb_policy_transform} becomes
\begin{align}
  u_\tau %-\beta \sigma^{-\alpha} u_x}
  &=\frac{1}{2}{(\sigma x + \beta \tau)}^2u_{xx}
    +\sigma (\sigma x+\beta \tau)u_x  % -\beta
    + uu_x,&\tau>0,\,x&> -\beta \tau/\sigma,
\end{align}
with matching conditions
\begin{equation}
  \lim_{x \to -\infty} u_x(\tau,x) = -1/\tau \qquad \textnormal{and}
  \qquad \lim_{x\to \infty}u(\tau,x) = 0.
\end{equation}
The leading order equation for $u = u^{(0)}+\sigma u^{(1)}+\sigma^2 u^{(2)}+\cdots$ is therefore
\begin{equation}
  u^{(0)}_\tau = \frac{1}{2}{(\beta \tau)}^2u^{(0)}_{xx} + u^{(0)}u^{(0)}_x.
\end{equation}
This equation has a similarity solution using the transformations
\begin{align}
  \eta(\tau,x)&= x/\tau^{3/2},\\
  u^{(0)}(\tau,x) &= \tau^{1/2}f(\eta(\tau,x)).
\end{align}
The function $f(\eta)$ must satisfy the boundary-value ODE
\begin{align}\label{eq:bvp_ode_leading}
  \beta^2f^{\prime\prime} + (3\eta + 2f)f^\prime - f
  &= 0,\\
  f&\to 0 &\textnormal{as } \eta&\to \infty,\\
  f^\prime &\to -1&\textnormal{as } \eta&\to -\infty.
\end{align}
\Cref{fig:inner_leading_f} shows numerically computed solutions to the
ODE for different values of $\beta$.
\begin{figure}[hbt]
  \centering
  \includegraphics[width=0.5\textwidth,axisratio=1.5]{inner_asymptotics_f}
  \caption[Numerically computed solutions of asymptotics problem]{Numerically computed solutions of~\eqref{eq:bvp_ode_leading}.
    Note the similarity to the optimal pricing function curves zoomed
    in near the kink in \Cref{fig:hjb_cec_zoom}.
  }\label{fig:inner_leading_f}
\end{figure}

We briefly note that we can find a closed form expression for the
second-order correction in the outer region $\xi < \beta t$. Let $\psi
= \psi^{(0)} + \sigma^2 \psi^{(1)} +
\sigma^4\psi^{(2)}+\cdots$, where $\psi^{(0)}$ is given
in~\eqref{eq:psi_leading} and we wish to find $\psi^{(1)}$.
By substituting this expansion into~\eqref{eq:hjb_policy_transform} and
matching the $\sigma^2$-terms we get the equation
\begin{align}
  \psi_\tau^{(1)}
  &= \psi_\xi^{(0)}(\psi^{(1)}+\xi) + (\psi^{(0)}- q_1)\psi_\xi^{(1)},\\
  \psi^{(1)}(\tau,0) &= 0.
\end{align}
When $\xi<\beta$ we have
\begin{align}
  \psi^{(0)}(\tau,\xi) &= q_1 - \xi/\tau,\\
  \psi_\xi^{(0)}(\tau,\xi) &= -1/\tau.
\end{align}
The PDE for $\psi^{(1)}$ is therefore
\begin{equation}
  - \tau \psi_\tau^{(1)} =\psi^{(1)} + \xi + \xi\psi_\xi^{(1)}.
\end{equation}
The stationary solution to this equation is
\begin{equation}
  \psi^{(1)}(\tau,\xi) = -\xi/2,
\end{equation}
which also satisfies the boundary condition.
Thus the optimal price in the region $\xi<\beta \tau$
that takes into account the uncertainty $\sigma$ reduces the price
from the deterministic-case solution,
\begin{equation}
  \psi(\tau,\xi)\approx \psi^{(0)}(\tau,\xi)-\sigma^2\xi/2.
\end{equation}
This explains the slightly lower prices observed from the comparison
between $a^D$ and the numerical approximation to $a^B$ in the plots of
\Cref{fig:hjb_cec_zoom}.



\subsection{Example simulation}
The asymptotic results together with \Cref{fig:hjbsol_a} indicate that
one can expect a price decrease over time when following the optimal
pricing policy $\alpha^B(t)$.  The numerical investigation in the next
example verifies this, but highlights that there are negligible gains
in total profit from pricing according to $\alpha^B(t)$ rather than
the deterministic-case pricing policy $\alpha^D(t)$.

\begin{figure}[htb]
  \centering
  \includegraphics{gbm_stats_a_diff}
  \caption[Statistics of the price paths from \Cref{ex:impact_uncertainty}]{Statistics of the price paths from
    \Cref{ex:impact_uncertainty}, %\Cref{ex:impact_uncertainty},
    showing the $0.05,0.5$, and
    $0.95$ quantiles of $\alpha(t)$. From top to bottom, the optimal
    policy, the deterministic heuristic, and their relative difference.
    The optimal policy starts slightly higher, and decreases over
    time.
    Note that the $0.05$ and $0.5$ quantiles lie on top of each
    other for $\alpha^D(t)$, so prices are likely to stay constant
    at $1/2$.
  }\label{fig:gbm_stats_a}
\end{figure}
\begin{example}\label{ex:impact_uncertainty}
  Let us simulate the system from
  \Cref{ex:function_impact_uncertainty}
  with the numerical approximation to $\alpha^B(t)$ and compare it to $\alpha^D(t)$.
  We draw \num{100000} sample paths from the underlying Brownian
  motion $W(t)$, and set the price to be constant on intervals of size $\Delta t = 0.01$
  using a policy $\alpha(t)$. We use the estimator described in
  \Cref{subsec:param_estim} as an approximation to $G(t)$.
  The simulations are run with both $\alpha^B(t)$ and $\alpha^D(t)$, and
  statistics of their paths are shown in \Cref{fig:gbm_stats_a}.
  The prices $\alpha^B(t)$ start slightly higher than $\alpha^D(t)$, but
  will then over time decrease, on average, towards the lower bound
  $\delta=1/2$. We also see that the deterministic heuristic is less
  anticipative, and will begin increasing the prices compared to the
  optimal policy after $t=0.2$.

  The measure of interest, however, is how much profit the different
  policies make. Recall that the profit of following a policy
  $\alpha(t)$ is the random variable
  \begin{equation}
    P(\alpha)=\int_0^{T_h}q(\alpha(u))\alpha(u)G(u)\,du-CS^\alpha(T).
  \end{equation}
  From simulations we estimate the distribution of $P(\alpha)$ for the
  optimal and deterministic policies, and compare their performance.
  The improvement is negligible, as we see from the relative statistics
  \begin{align}
    \mathbb{E}\left[1-P(\alpha^D)/P(\alpha^B)\right]
    &\approx 2\times 10^{-4},\\
    \mbox{std}\left[1-P(\alpha^D)/P(\alpha^B)\right]
    &\approx 6\times 10^{-4},\\
    \mbox{Median}\left[1-P(\alpha^D)/P(\alpha^B)\right]
    &\approx -1\times 10^{-4}.
  \end{align}
  The calculated optimal pricing policy results in \SI{0.02}{\percent} higher profits
  than the heuristic on average. It even results in lower profits than the
  heuristic for more than \SI{50}{\percent} of the realisations.
  \Cref{fig:gbm_histogram_a_cec_hjb} shows a histogram that approximates
  the distribution of the relative loss from using $\alpha^D(t)$
  over the optimal pricing policy. The differences between the two are
  small, but the distribution is non-symmetric:
  The heuristic $\alpha^D(t)$ results in slightly larger profit for more
  than half of
  the realisations, at the expense of performing  worse for the remaining realisations.
  \begin{figure}[htb]
    \centering
    \includegraphics[width=0.6\textwidth,axisratio=1.25]{gbm_histogram_a_cec_hjb}
    \caption[Profit sample distribution comparing the deterministic-case
    pricing heuristic to the Bellman policy]{The histogram shows the distribution of the relative
      profit by using the deterministic pricing heuristic $\alpha^D(t)$
      to using the optimal policy $\alpha^B(t)$, as described in
      \Cref{ex:impact_uncertainty}. %\Cref{ex:impact_uncertainty}.
      Positive values
      correspond to realisations of $W(t)$ where
      $\alpha^B(t)$ is better.
      The shape of the distribution is similar to the discrete-time pricing problem
      in \Cref{ch:discrete_control}.
    }\label{fig:gbm_histogram_a_cec_hjb}
  \end{figure}
\end{example}

In \Cref{ex:impact_uncertainty} and
\Cref{fig:gbm_histogram_a_cec_hjb} it appears that the relative
improvement in profits by pricing according to $\alpha^B(t)$, rather than the heuristic
policy $\alpha^D(t)$, is negligible.
Further numerical experiments for other values of $\sigma$ strengthen
these results. See \Cref{tbl:profit_hjb_cec_statistics} for summary
statistics of the relative difference in profits
$1-P(\alpha^D)/P(\alpha^B)$.
The relative improvement of following the strategy $\alpha^B(t)$
increases with $\sigma$, however the standard deviations are all on
the order of \SIrange{0.01}{0.1}{\percent}.
\begin{table}[htb]
  \centering
  \begin{tabular}{lrrrrr}
    \toprule
    $\sigma$ & mean & std & $\mathcal{Q}_{0.05}$ & $\mathcal{Q}_{0.5}$ & $\mathcal{Q}_{0.95}$\\
    \midrule
    0.05& 0.0 & 0.4 &-0.5&-0.2&0.9\\
    0.1&0.2& 0.6&-0.4&-0.1&1.4\\
    0.2&0.3& 0.7&-0.3&0.0&1.8\\
    0.4&0.4& 1.2&-0.3&0.1&1.8\\
    \midrule
             & $\times 10^{-3}$&$\times 10^{-3}$&$\times 10^{-3}$&$\times 10^{-3}$&$\times 10^{-3}$\\
    \bottomrule
  \end{tabular}
  \caption[Profit statistics comparing the Bellman and
  deterministic-case policies]{Summary statistics of the relative profit difference
    $1-P(\alpha^D)/P(\alpha^B)$ from the model in
    \Cref{ex:impact_uncertainty} with different levels of
    uncertainty $\sigma$.
    The headings $\mathcal{Q}_z$ denote the $z$-quantile of the distribution.
  }\label{tbl:profit_hjb_cec_statistics}
\end{table}

\section{Extensions of the pricing problem}\label{sec:extensions}
For our one-product system with multiplicative parameter dynamics with uncertainty,
two natural extensions to the pricing problem are
\begin{enumerate}
\item to incorporate
  other forms of uncertainty, and
\item to formulate the problem for risk-averse
  decision makers.
\end{enumerate}
For example, the costs of handling unsold stock may
not be known a priori, and risk aversion may be modelled from an
expected utility viewpoint.
These extensions increase the input dimensions of the corresponding
HJB equations, which we present without further discussion.


\subsection{Other forms of uncertainty}
In the prior sections, the source of randomness in the system has come from
the multiplicative term $G(t)$, modelled as a geometric Brownian
motion martingale. A different non-negative stochastic process may be more appropriate, and the
choice of dynamics can be guided by existing sales data.
% If there is
% an assumption that parameters in the demand function has a true
% underlying value constant in time, then a mean-reverting process can
% be more appropriate. For example, say $g^\dagger$ is the true value,
% but stochasticity in the system means that the multiplicative term
% deviates from this value. For some constant $b>0$
% and a function $\sigma(t,g)\geq 0$, such a multiplicative term can be
% modelled as the solution $G_m(t)$ to the SDE
% \begin{equation}
%   dG_m(t)=b\cdot(g^\dagger-G_m(t))\,dt+\sigma(t,G_m(t))\,dW(t).
% \end{equation}
More generally, the demand function $q(a)$ depends on multiple
parameters that exhibit different levels of
uncertainty and dynamics in time. Another parameter in the pricing
problem is the unit cost $C$, where its value at the terminal time may
depend on factors unknown at times $t<T$.
Let $\theta(t)\in\mathbb{R}^n$ denote the vector of parameters that are
relevant to the problem, and say the demand function  and the unit cost
depend explicitly on $\theta$. We write $q(a;\theta)$ and $C(\theta)$
for this dependence.
Within the diffusion based stochastic framework, we can model the
dynamics of $\theta(t)$ with functions $b(t,\theta)\in\mathbb{R}^n$,
$\sigma(t,\theta)\in\mathbb{R}^{n\times p}$, and a vector-valued, uncorrelated,
Brownian motion $W(t)\in\mathbb{R}^p$.
We assume that $\theta(t)$ does not depend on the pricing policy
$\alpha(t)$ or the remaining
stock $S^\alpha(t)$. Thus,
the system for the pricing problem is described by the SDE,
\begin{align}
  d\theta(t)&=b(t,\theta(t))\,dt + \sigma(t,\theta(t))\,dW(t),\\
  dS^\alpha(t)&=-q(\alpha(t);\theta(t)).
\end{align}

Let $\Theta\in\mathbb{R}^n$ denote the state space for
$\theta(t)$.
Let $\nabla_\theta v$ and $\Hess{}_\theta v$ denote the gradient and Hessian of
$v(t,s,\theta)$ with respect to $\theta$. We denote the transpose
operator with a superscript $\intercal$, and introduce the
volatility matrix $\Sigma(t,\theta) =
\sigma(t,\theta)\Transpose{{\sigma(t,\theta)}}$.
The HJB approach for the pricing problem is then to find
$v:{[0,1]}^2\times\Theta\to\mathbb{R}$ which satisfies
\begin{align}
  v_t+\Transpose{{b(t,\theta)}} \nabla_\theta v
  + {\textstyle\frac{1}{2}}\mbox{tr}\left( \Sigma(t,\theta)
  \Hess{}_\theta v \right)
  +\max_{a}\{q(a;\theta)(a-v_s)\}&=0,\\
  v(T,s,\theta) &= -C(\theta)s.
\end{align}
Additional boundary conditions may be necessary, depending on
$\theta(t)$. For example, the boundary condition for $g=0$ in our
multiplicative model from the previous sections.
The value function and the pricing function now depend
explicitly on each element in $\theta$, thus increasing the dimension
of the corresponding HJB equation. Efficient algorithms for solving
high-dimensional PDEs of this form exist. See, for example, the
multigrid preconditioning approach by \citet{reisinger2017boundary}, or
the splitting into a sequence of lower-dimensional PDEs by \citet{reisinger2017finite}.
In higher dimensions it is of even greater importance to critically
balance computational cost with the suboptimality of
approximations. As \Cref{sec:stochastic_hjb} shows, approximate
policies can perform well. This underscores the importance of assessing whether
new sources of randomness in the model have significant impact on the
objective and the optimal policy.


\subsection{Expected utility risk aversion}
One may argue that for a retailer as a whole, an assumption of a
risk-neutral decision maker is valid. For individual product managers,
whose performance is evaluated over shorter time horizons, a degree of
risk aversion can be preferable from their point of view. Formulations and investigations of the
impact of risk aversion on pricing policies is also noted as an
interesting area of research by \citet{bitran2003overview}.
For investigations of the expected utility problem with Poisson process
demand, see, for example, \citet{lim2007relative} or \citet{feng2008risk}.

In this section, we assume that the decision maker is evaluated based on
the performance of the total profit from selling a product over the
time interval $[0,T]$. Then, the pricing decision at time $t$ may also depend
on how much revenue has been accrued at that time. So we
introduce a state variable $R^\alpha(t)$ representing the accrued
revenue at time $t$, with dynamics
$dR^\alpha(t)=\alpha(t)q(\alpha(t);\theta(t))\,dt$.
We consider risk aversion based on
an expected utility-maximising decision maker.
Given a utility function $U(x)$, the pricing problem is then to
find a pricing policy $\alpha$ that maximises the expected utility
$\mathbb{E}[U(R^\alpha(T)-C(\theta(T))S^\alpha(T))]$.
For this utility function, the value function is defined as
\begin{equation}
  v(t,s,\theta,r)=
  \max_{\alpha\in\mathcal{A}}
  \mathbb{E}_t\left[
    U\Bigl(R^\alpha(T)-C\bigl(\theta(T)\bigr)S^\alpha(T)\Bigr)
  \right].
\end{equation}
The corresponding HJB problem is then to find
$v:{[0,T]}\times{[0,1]}\times \Theta\times{[0,\infty)}\to\mathbb{R}$ that solves
\begin{align}\label{eq:hjb_utility}
  v_t+\Transpose{{b(t,\theta)}} \nabla_\theta v
  + {\textstyle\frac{1}{2}}\mbox{tr}\left( \Sigma(t,\theta)
  \Hess{}_\theta v \right)
  +\max_{a}\{q(a;\theta)(av_r-v_s)\}=0&,\\
  v(T,s,\theta,r) = U(r-C(\theta)s)&,
\end{align}
plus additional boundary conditions. The impact of the risk aversion
on pricing decisions appears in the maximisation term in the HJB
equation,
\begin{equation}
  \max_a\{q(a;\theta)(av_r-v_s)\}.
\end{equation}
The $v_r$ term represents
the relative importance of accruing more revenue to the utility of
selling more stock, as represented by the $v_s$ term.
The risk-neutral case $U(x)=x$ that we have considered in the previous
sections give $v_r=1$.
% A different interpretation of the effect of
% $v_r$ is as a taxation
% of revenue, where $q(a;\theta)av_r$ represents the monetary value
% after taxation (with the weird concept that ).

% More insight can be made by looking at the evolution of the certainty
% equivalent value of the terminal profit when following the optimal pricing policy.
% For our purposes, the certainty equivalent utility of a random
% variable $X$ can be defined as
% $U^{-1}[\mathbb{E}[U(X)]]$, when $U$ is invertible. We thus define the certainty equivalent
% value function, given state $(t,s,\theta,r)$, as
% \begin{equation}
%   e(t,s,\theta,r) = U^{-1}(v(t,s,\theta,r)).
% \end{equation}
% The terminal condition for this function is $e(T,s,\theta,r)=r-Cs$.
% All derivatives of $v$ can then be written in terms of $e$
% using the chain rule:
% \begin{align}
%   \nabla_zv &= U'(e)\nabla_ze,& \text{where }
%   &z\in\{t,s,\theta,r\},\\
%   D^2_\theta v&=U'(e)D^2_\theta e
%   +U''(e)[\nabla_\theta e]\Transpose{{[\nabla_\theta e]}}.
% \end{align}
% Before we show the resulting PDE for $e(t,s,\theta,r)$, it is useful
% to introduce the concept of Absolute Risk Aversion (ARA), as defined
% by \citet{pratt1964risk}. For twice differentiable utility functions,
% we denote the ARA coefficient $\mathcal{X}(x)=-U''(x)/U'(x)$ at wealth $x$
% as a way of measuring the risk
% aversion of the decision maker.
% For strictly increasing utility functions $U(e)$, we can substitute
% the expressions for the derivatives of $v$
% into~\eqref{eq:hjb_utility}, and divide by
% $U'(e)>0$ to obtain
% \begin{align}\label{eq:hjb_ceutility}
%   e_t
%   +\Transpose{{b(t,\theta)}} \nabla_\theta e
%   + {\textstyle\frac{1}{2}}\mbox{tr}\left( \Sigma(t,\theta)
%     \Hess{}_\theta e \right)
%   +\max_{a}\{q(a;\theta)(ae_r-e_s)\}
%   &=
%   {\textstyle\frac{1}{2}}\mathcal{X}(e)\,\mbox{tr}\left( \Sigma(t,\theta)
%     [\nabla_\theta e]\Transpose{{[\nabla_\theta e]}} \right),\\
%   e(T,s,\theta,r) &= r-C(\theta)s,
% \end{align}
% plus additional boundary conditions.
% We see that $e(t,s,r,\theta)$ is
% therefore evolving in time in a similar manner to the risk-neutral
% value function $v(t,s,r,\theta)$ in the previous sections, with an
% additional term that depends on the decision maker's risk aversion
% $\mathcal{X}(e)$.
% Note that $\mathcal{X}(x)$ is always negative, zero, and positive for a
% risk-averse, risk-neutral, and risk-seeking decision maker respectively.
% In addition, we know that $\mbox{tr}(\Sigma(t,\theta)[\nabla_\theta e]\Transpose{{[\nabla_\theta e]}})\geq
% 0$, since the matrix in the trace is positive semi-definite.
% This gives us the following two, expected, insights:
% \begin{itemize}
% \item The certainty equivalent value decreases with increasing
%   risk aversion, but the effect diminishes as we get closer to the
%   terminal time.
% \item Uncertainty in the system can have an order one impact on the
%   certainty equivalent value if $\mathcal{O}(\mathcal{X}(e))=\mathcal{O}
%   \left(\rho\left( {\Sigma(t,\theta)}^{-1}
%     \right)\right)$, where $\rho(M)$ denotes the spectral radius of a matrix.
% \end{itemize}

% We finish the section with an  example of a particular family of utility functions, and
% discuss the resulting pricing policy's dependence on revenue.
% \begin{example}[Constant Absolute Risk Aversion]
%   Consider utility functions with constant ARA, $\mathcal
%   X(x)=\gamma>0$, of the form $U(x)=-e^{-\gamma x}$.
%   For such utility functions, the optimal price is independent of the
%   accrued revenue.
%   For brevity, we omit the dependence of $\theta$ on the demand
%   function $q(a)$ and the unit cost $C$.
%   First, note that if $R^\alpha(t)=r$, then
%   $R^\alpha(T)=r+\int_t^{T_h}q(\alpha(u))\alpha(u)\,du$.
%   Second, note that
%   $U(R^\alpha(T)-CS^\alpha(T)) =  e^{-\gamma r}
%   U\left( \int_t^{T_h}q(\alpha(u))\alpha(u)\,du-CS^\alpha(T) \right)$.
%   The certainty equivalent value function that satisfies the
%   PDE~\eqref{eq:hjb_ceutility} can be written as
%   \begin{align}
%     e(t,s,\theta,r)
%     &=-\frac{1}{\gamma} \log
%     \left\{ -\max_{\alpha \in\mathcal{A}} \mathbb{E}_t
%       \left[ U\left( R^\alpha(T)-CS^\alpha(T) \right)\right]\right\} \\
%     &= r - \frac{1}{\gamma}\log
%     \left\{ -\max_{\alpha \in\mathcal{A}} \mathbb{E}_t
%       \left[U\left( \int_t^{T_h}q(\alpha(u))\alpha(u)\,du-CS^\alpha(T) \right)\right]\right\}.
%   \end{align}
%   There is no explicit dependence on $r$ in the maximisation term, and
%   thus $\alpha(t)$ will not depend on $R^\alpha(t)$.
%   This result is to be expected, since the Constant Absolute Risk
%   Aversion utility represents a decision maker that does not change
%   their level of risk aversion based on the amount of current wealth.
% \end{example}

% In comparison to the risk-neutral system considered in the previous
% sections, higher risk aversion
% would lead to leaving the price the same in the region where we are not expected
% to sell out of the product, and increasing the price when we expect to sell
% out of the product.

\section{Discussion}\label{sec:cts_conclusion}
This chapter focuses on continuum approximations for dynamic
pricing problems under uncertainty.
Most of the existing literature on continuous time dynamic pricing for
revenue management is based on Poisson
processes. This is suitable for many applications,
however, for large retailers approximating the number of sales and
stock as a continuum can simplify calculation of pricing rules.
We present an approach for modelling the sales as a continuous time
dynamical system, where the uncertainty in demand arises from
stochastic processes.
An advantage over the Poisson process model is that this approach
allows us to directly model the demand volatility.
Under this model, we consider a pricing problem where the retailer
aims to deplete  inventory of a product at maximum profit.
By formulating the problem as a stochastic optimal control problem,
we can express the optimal pricing policy in terms of the solution to
a nonlinear PDE.\@
For linear and exponential demand functions, we find closed-form
expressions for the pricing policy when the system is deterministic.
It turns out that, for a risk-neutral decision maker, the
deterministic pricing policy is a near-optimal heuristic for systems
with demand uncertainty.
Numerical errors in calculating the optimal pricing policy
may, in fact, result in lower profits on average than with the heuristic
pricing policy.
% If the decision maker is highly risk averse, we show that this will
% have a significant impact on the perceived utility of the profits,
% and potentially reduce the applicability of the pricing heuristic.

There are two topics of particular interest for future study.
The first is to understand why demand uncertainty has such a
small effect on the optimal pricing policy for risk-neutral decision
makers, and whether constraints such as requiring monotone-in-time pricing
policies may increase this effect. Second, a case study of the continuum
model framework for multiple products and time-dependent demand is
needed, in
order to understand how well this approach can scale to revenue
management implementations for retailers.




\biblio{} % Bibliography when standalone
\end{document}

%%% Local Variables:
%%% mode: latex
%%% TeX-master: t
%%% TeX-engine: luatex
%%% TeX-command-extra-options: "--shell-escape"
%%% End:
